\newcommand{\theauthor}{Hilmar Gústafsson}
\newcommand{\thesupervisor}{Ivan Aaen}
\newcommand{\thetitle}{Using Open Source for Privacy}
\newcommand{\thesubtitle}{Porting Peer-to-Peer Matrix to Android}
\newcommand{\thedate}{\today}
\newcommand{\theprojecttype}{Semester Project}
\newcommand{\thegroup}{SW816F20}
\newcommand{\thetheme}{Mobility}
\newcommand{\theabstract}{Privacy is becoming harder and harder to maintain as software becomes more intrusive.
	Walled gardens are closed platforms which threaten the privacy of their users, and the diversity that the internet makes possible.
	The Matrix Standard is an attempt to liberate our communication and privacy through open efforts.
	We investigate the viability of porting Peer-to-Peer Matrix to Android, as a method of improving personal privacy.
	We contribute to the Dendrite open source project, and design and implement modifications to an Android application, so that it also runs a Dendrite server.
	The resulting application is capable of communicating via Peer-to-Peer networking, and is 34\% larger than the unmodified application, or 105MB in all.
	We determine that it is feasible to run Peer-to-Peer Matrix on Android given that room sizes and media are limited.
}