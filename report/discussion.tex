\chapter{A Critical Evaluation}\label{chp:a_critical_evaluation}
Earlier in the paper, we decided to evaluate the implementation based on two mobile constraints, namely battery life and storage space.
To do this, we the sections by placing expectations based on the design and implementation.
Afterwards, we measure the difference using the tools available in the Android platform.

\section{Battery Life}\label{sec:battery_life}
\todo{Write this when the data from Battery Historian is ready.}

\subsection{Further Optimizations}
However, the energy impact could be reduced even further by explicitly pausing Dendrite when the app is paused.
This would be possible due to two reasons:
\begin{enumerate}
	\item{
	      \textbf{Android Application Lifecycle}:
	      The \texttt{Application} interface is similar to the \texttt{Callback} interface, in the sense that it allows us to receive information from a separate process.
	      In this case, the Android operating system can notify us of different events occurring in the app if we implement its \texttt{Application} interface.
	      Two examples of these events are \texttt{onPause} and \texttt{onResume}, which are called by the operating system when the app is paused and resumed.
	      We could use these lifecycle events to properly pause and resume the Dendrite server.
	      }
	\item{
	      \textbf{Java \texttt{Thread}s}:
	      Dendrite is started in its own thread using the Java \texttt{Thread} \ac{API}\@.
	      Creating a new task creates an object which we can use to suspend and resume the task running on the thread.
	      }
\end{enumerate}

These optimizations are useful, but not necessary, as the Android operating system is quite good at suspending tasks on its own.

\section{Storage Space}\label{sec:storage_space}
The space used by an application is split into three categories:
\begin{enumerate}
	\item{
	      \textbf{App size}:
		  App size is determined by the space taken by the application itself, which includes Java class files, resource files and a single file which contains compiled resources\cite{android_devdocs_apksize}.
	      }
	\item{
	      \textbf{User data}:
		  Each Android application has access to a special folder where it can store user data.
		  User data is meant for more permanent types of data, such as credentials and settings.
	      }
	\item{
	      \textbf{Cache}:
		  Each Android application is also given a storage mechanism known as the cache.
		  The cache is meant for temporary data, such as resources, which it can download to avoid excessive network use.
	      }
\end{enumerate}

The storage space of the application varies from device to device.
We use a device with the arm64 architecture, on which Riot for Android has an App size of 78.38MB\@.
Since the design uses a mostly unmodified Riot with an Embedded Dendrite, we can expect this size to increase.
Additionally, Dendrite stores information about registered users, rooms, other servers and message metadata.
The cache and user data are therefore expected to increase significantly as well.
In the best case scenario, the size increase is only acceptable on more powerful Android devices.

\subsection{Storage Use}\label{sec:storage_use}
When we compile the Dendrite server for our phone, the generated \texttt{server.aar} artifact has a size of 19.5MB\@.
The app size is a total of 105MB\@, which is an increase of 34\%, or 26.7MB\@.

This increase may seem conservative, but the main impact of the application occurs in the permanent storage of message metadata on the phone instead of a dedicated server.

Matrix rooms do not have an upper limit on the amount of participants. On a device with less storage space, joining just one of those rooms could fill up the entire device.
On the other hand, text can be stored in a very compact way compared to media formats, which means that this is not necessarily the case, even for the larger rooms, unless they are media heavy.
Therefore, the storage space used by the app will be reasonable as long as the user avoids using excessive media and limits the size of chat rooms.


\chapter{Open Source, Complexity and Alternatives}\label{chp:open_source_complexity_alternatives}
This project was intentionally conducted with \ac{OSS}.
We are concerned with privacy and individual freedom, which we believe \ac{OSS} helps preserve.
However, there is an established tradition of how one conducts oneself in the \ac{OSS} environment, and the projects in which we participate are large and complex.
\ac{OSS} naturally leads to sharing and modifications, which may result in alternative implementations of a standard or protocol.
We wish to discuss how these elements affect our work.

\section{Open Source Software}\label{sec:open_source}
The term ``Open Source Software'' should be understood as software which is licensed by a license recognized by the \ac{OSI}\cite{open_source_initiative}.
In a nutshell, these licenses allow the software to be used, modified and redistributed, although the terms for those actions vary.
As mentioned in the beginning of the chapter, we are strong believers in the principles of open source, and will in this section describe how it has affected the project.

\subsection{Benefits of Open Source}\label{subsec:benefits_of_open_source}
This project uses code which is completely open source.
Specifically, the code is located in the following repositories:
\begin{itemize}
	\item{
		\github{LiHRaM/riot-android}:
		The modified Riot app which runs Dendrite as a peer-to-peer server.
	}
	\item{
		\github{LiHRaM/server}:
		The Go library which is compiled to \texttt{server.aar} and used in Riot for Android.
	}
	\item{
		\github{matrix-org/dendrite}:
		We have contributed to the official Dendrite, where our submissions were received as \ac{OSS}.
	}
\end{itemize}

The main purpose for storing the code in publicly available repositories is the \textit{free sharing of information}.
Anyone can learn from the software, is free to use it, and could even improve it.
Contrast this with closed software, where the impact of discoveries is limited to people who have been given access to the information.

The benefits of free sharing of information could be broken down into the following benefits:
\begin{itemize}
	\item{
		\textbf{Open Collaboration}:
		Anyone who is willing to contribute to open source may do so, given that they adhere to the terms of the licenses.
		Due to this, we were able to freely collaborate with the Dendrite developers to experiment with peer-to-peer Matrix.
	}
	\item{
		\textbf{Software Quality}:
		We have learned much from the software released as \ac{OSS} in this project.
		We believe that other students and those who wish to learn have a greater chance of becoming quality engineers by reading more quality software.
	}
	\item{
		\textbf{Security}:
		In order to construct a fully safe product, the author must be able to understand all of its components.
		By using only open source software, authors can audit each component for security concerns.
		We have not performed an audit of the components in Matrix, but others have\todo{citations}.
	}
\end{itemize}

\subsection{Disadvantages of Open Source}\label{subsec:open_source_disadvantages}
In our experience, \ac{OSS} development has drawbacks, which are mostly caused by the diversity of collaborators.
The priorities of the participants in an \ac{OSS} project vary greatly, as do their motivations for contributing to the software.
Our project is a University experiment, whereas the main Dendrite contributors are paid to contribute.
We summarize the main drawbacks which we experienced in the following list:
\begin{itemize}
	\item{
		\textbf{Communication}:
		Asynchronous communication is useful for large, distributed projects, but needs to be carefully moderated.
		Some of the feedback that we received on our initial contributions felt hostile, for example, which is not helpful, and often harmful.
	}
	\item{
		\textbf{Priorities}:
		We were lucky that our intent to experiment Peer-to-Peer Matrix happened to be aligned with the Matrix team.
		However, the Matrix team wished to implement it via the browser, whereas we intended to do it via Mobile.
		Because of this, we had to do extra work to get the mobile version working, which resulted in a less complete implementation.
	}
	\item{
		\textbf{Activity}:
		The different software projects we used were at different levels of maintenance, with Riot for Android being mostly abandoned, and Dendrite being very new and active.
		The non-official implementations were also at varying levels of activity, with some being abandoned, and others not being usable yet.
	}
	\item{
		\textbf{Funding}:

	}
\end{itemize}

Overall, working with \ac{OSS} was a positive experience in this project.
When working with open source, we believe it is important to be aware of the risks and drawbacks, and to include them when considering the use of \ac{OSS} in projects.

\section{Code Complexity}\label{sec:code_complexity}
\todo{%
	Discuss the process of getting to know a larger code base, and the risks involved.
	How do you prioritize which things you learn, etc?
}


\section{Alternative Solutions}\label{sec:alternative_solutions}
\todo{Describe the possibility of developing a client and a server which are more thoroughly integrated, for example by skipping the messaging via HTTP.}
\todo{Describe the potential use of the WebAssembly solution via a something like React Native, etc.}
\todo{Describe potential alternatives to libp2p.}

\subsection{Fog Computing}\label{sec:fog_computing}
Since running a modified homeserver on the phone has drawbacks, especially with regard to storage, using fog computing might be a useful alternative.
In practice, this would mean providing a single homeserver locally for one or more clients, which would handle the storage and Server-Server \ac{API}.
A paper from 2015 by Hassan et al indicates that this form of computation may be more effective than offloading completely to a central server\cite{hassan2015help}.