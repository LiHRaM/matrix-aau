\chapter{A Critical Evaluation}\label{chp:a_critical_evaluation}
Earlier in the paper, we decided to evaluate the implementation based on two mobile constraints, namely battery life and storage space.
To do this, we begin by placing expectations based on the design and implementation.
Afterwards, we measure the difference using the tools available in the Android platform.

\section{Battery Life}\label{sec:battery_life}
We expect an increase of power usage after our modifications.
Although there should be minimal delay in the Client-Server communication, the requests are still made via HTTP\@.
In addition to the Client-Server communication, the application now also handles Server-Server communication and federation, as well as peer discovery.

\subsection{Tracking Power Use}
We use the program recommended by the Google Developer documentation, Battery Historian\cite{battery_historian}.
The approach is explained in the documentation, and can be summarized as the following list of steps:
\begin{enumerate}
	\item{
	      \textbf{Optional: Reset Battery Stats}:
	      Android phones constantly monitor battery usage in great detail.
	      In order to reduce noise, it is recommended to reset this data just before using the desired application.
	      }
	\item{
	      \textbf{Use Application}:
	      While using the application, the Android phone monitors the events and their effects on the power level of the phone.
	      }
	\item{
	      \textbf{Store Bug Report}:
	      After using the app, the data can be transferred to a computer in the form of a bug report.
	      }
	\item{
	      \textbf{Upload Bug Report}:
	      Battery Historian is actually a web server which allows uploading bug reports.
	      Once the bug report is loaded, Battery Historian presents an in-depth dashboard for analyzing the data in the bug report.
	      }
	\item{
	      \textbf{Analyze Data}:
	      There are many ways to analyze the available data, including a way to measure the battery impact of a single application.
	      }
\end{enumerate}

We applied these steps to two phones, where one phone had the unmodified Riot app, and the other had the peer-to-peer Riot app which we implemented.
The peer-to-peer application participated in a small conversation between two individuals, whereas the original app accessed rooms on the Matrix network.

The results showed that the modified app is more active in the background, see Table~\ref{tab:battery_comparison}.
Presumably, this is because the Dendrite server constantly probes for peers on the network, whereas the original application only wakes on push notifications.
In the foreground, the original app consumed more power.
We suspect, however, that the power consumption of the modified app does not scale well with chatroom sizes and media use, as it is processed by both the homeserver and the client.

These results show that the battery consumption of peer-to-peer Riot is reasonable, given constrained use, such as small chatroom sizes.


\begin{table}[]
	\centering
	\begin{tabular}{@{}lcc@{}}
		\toprule
		                           & \multicolumn{1}{l}{\textbf{Peer-to-Peer Riot}} & \multicolumn{1}{l}{\textbf{Unmodified Riot}} \\ \midrule
		Estimated Power Use / Hour & 6.9\%                                          & 13.7\%                                       \\
		Background Packets / Hour  & 1620                                           & 0                                            \\ \bottomrule
	\end{tabular}
	\caption{A comparison of battery use of two Riot apps. The Peer-to-Peer application is more active in the background, whereas the original consumes more power in the foreground.}%
	\label{tab:battery_comparison}
\end{table}


\subsection{Further Optimizations}
We did not perform further optimizations after this analysis, but the background activity could be disabled.
This would be possible due to two reasons:
\begin{enumerate}
	\item{
	      \textbf{Android Application Lifecycle}:
	      The \texttt{Application} interface is similar to the \texttt{Callback} interface, in the sense that it allows us to receive information from a separate process.
	      In this case, the Android operating system can notify us of different events occurring in the app if we implement its \texttt{Application} interface.
	      Two examples of these events are \texttt{onPause} and \texttt{onResume}, which are called by the operating system when the app is paused and resumed.
	      We could use these lifecycle events to properly pause and resume the Dendrite server.
	      }
	\item{
	      \textbf{Java \texttt{Thread}s}:
	      Dendrite is started in its own thread using the Java \texttt{Thread} \ac{API}\@.
	      Creating a new task creates an object which we can use to suspend and resume the task running on the thread.
	      }
\end{enumerate}

\section{Storage Space}\label{sec:storage_space}
The space used by an application is split into three categories:
\begin{enumerate}
	\item{
	      \textbf{App size}:
	      App size is determined by the space taken by the application itself, which includes Java class files, resource files and a single file which contains compiled resources\cite{android_devdocs_apksize}.
	      }
	\item{
	      \textbf{User data}:
	      Each Android application has access to a special folder where it can store user data.
	      User data is meant for more permanent types of data, such as credentials and settings.
	      }
	\item{
	      \textbf{Cache}:
	      Each Android application is also given a storage mechanism known as the cache.
	      The cache is meant for temporary data, such as resources, which it can download to avoid excessive network use.
	      }
\end{enumerate}

The storage space of the application varies from device to device.
We use a device with the arm64 architecture, on which Riot for Android has an App size of 78.38MB\@.
Since we do nothing to reduce the size of the application, we expect the app size to increase when we add the homeserver.
Additionally, Dendrite stores information about registered users, rooms, other servers and message metadata.
The cache and user data are therefore expected to increase.

\subsection{Storage Use}\label{sec:storage_use}
When we compile the Dendrite server for our phone, the generated \texttt{server.aar} artifact has a size of 19.5MB\@.
The app size is a total of 105MB\@, which is an increase of 34\%, or 26.7MB\@.

This increase may seem conservative, but the main impact of the application occurs in the permanent storage of message metadata on the phone instead of a dedicated server.

Matrix rooms do not have an upper limit on the amount of participants. On a device with less storage space, joining just one of those rooms could fill up the entire device.
On the other hand, text can be stored in a very compact way compared to media formats, which means that this is not necessarily the case, even for the larger rooms, unless they are media heavy.
Therefore, the storage space used by the app will be reasonable as long as the user avoids using excessive media and limits the size of chat rooms.

These policies can be enforced by placing limits on room sizes and disabling media in the embedded homeserver.


\chapter{Open Source, Complexity and Alternatives}\label{chp:open_source_complexity_alternatives}
This project was intentionally conducted with \ac{OSS}.
We are concerned with privacy and individual freedom, which we believe \ac{OSS} helps preserve.
However, there is an established tradition of how one conducts oneself in the \ac{OSS} environment, and the projects in which we participate are large and complex.
\ac{OSS} naturally leads to sharing and modifications, which may result in alternative implementations of a standard or protocol.
We wish to discuss how these elements affect our work.

\section{Open Source Software}\label{sec:open_source}
The term ``Open Source Software'' should be understood as software which is licensed by a license recognized by the \ac{OSI}\cite{open_source_initiative}.
The \ac{OSI} recognizes all licenses which comply with the \ac{OSD}\cite{open_source_definition}.
We believe in the general principles of open access to information, and would like to discuss our experiences working with \ac{OSS} in this project.

\subsection{Licensing}\label{subsec:licensing}
The Dendrite and Riot for Android projects use the Apache License 2.0\cite{apache_2pt0}.
These projects follow the policy that all contributions have to be signed off by the contributors, by which they accept the terms of contributing.
The contributions therefore cease to be our intellectual property, and become the property of the owner of the license, which in this case is the Matrix Foundation.

It can be risky to give work away in such a manner.
Especially so when the license does not prevent derivative works to be sold for profit, or for the source code to become unavailable, i.e.~closed source.
The Matrix Foundation was established as a neutral custodian of the Matrix Standard in order to avoid issues that might arise from actors working in their own interest, and is supposed to operate ``for the greater benefit of the whole ecosystem'', which is defined as anyone who uses the Matrix protocol\cite{matrix_org_foundation}.
It has a mission, a manifesto, and rules by which it must operate.

\subsection{Benefits of Open Source}\label{subsec:benefits_of_open_source}
This project uses code which is completely open source.
Specifically, the code is located in the following repositories:
\begin{itemize}
	\item{
	      \github{LiHRaM/riot-android}:
	      The modified Riot app which runs Dendrite as a peer-to-peer server.
	      }
	\item{
	      \github{LiHRaM/server}:
	      The Go library which is compiled to \texttt{server.aar} and used in Riot for Android.
	      }
	\item{
	      \github{matrix-org/dendrite}:
	      We have contributed to the official Dendrite, where our submissions were received as \ac{OSS}.
	      }
\end{itemize}

The main purpose for storing the code in publicly available repositories is the \textit{free sharing of information}.
Anyone can learn from the software, is free to use it, and could even improve it.
Contrast this with closed software, where the impact of discoveries is limited to people who have been given access to the information.

The benefits of free sharing of information could be broken down into the following benefits:
\begin{itemize}
	\item{
	      \textbf{Open Collaboration}:
	      Anyone who is willing to contribute to open source may do so, given that they adhere to the terms of the licenses.
	      Due to this, we were able to freely collaborate with the Dendrite developers to experiment with peer-to-peer Matrix.
	      }
	\item{
	      \textbf{Software Quality}:
	      We have learned much from the software released as \ac{OSS} in this project.
	      We believe that other students and those who wish to learn have a greater chance of becoming quality engineers by reading more quality software.
	      }
	\item{
	      \textbf{Security}:
	      In order to construct a fully safe product, the author must be able to understand all of its components.
	      By using only open source software, authors can audit each component for security concerns.
	      We have not performed an audit of the components in Matrix, but others have\cite{matrix_olm_cryptographic_review}.
	      }
\end{itemize}

\subsection{Disadvantages of Open Source}\label{subsec:open_source_disadvantages}
In our experience, \ac{OSS} development has drawbacks, which are mostly caused by the diversity of collaborators.
The priorities of the participants in an \ac{OSS} project vary greatly, as do their motivations for contributing to the software.
Our project is a University experiment, whereas the main Dendrite contributors are paid to contribute.
We summarize the main drawbacks which we experienced in the following list:
\begin{itemize}
	\item{
	      \textbf{Communication}:
	      Asynchronous communication is useful for large, distributed projects, but needs to be carefully moderated.
	      Some of the feedback that we received on our initial contributions felt hostile, for example, which is not helpful, and often harmful.
	      }
	\item{
	      \textbf{Priorities}:
	      We were lucky that our intent to experiment with Peer-to-Peer Matrix happened to be aligned with the Matrix team.
	      However, the Matrix team wished to implement it via the browser, whereas we intended to do it via Mobile.
	      Because of this, we had to do extra work to get the mobile version working, which resulted in a less complete implementation.
	      }
	\item{
	      \textbf{Activity}:
	      The different software projects we used were at different levels of maintenance, with Riot for Android being mostly abandoned, and Dendrite being very new and active.
	      The non-official implementations were also at varying levels of activity, with some being abandoned, and others not being usable yet.
	      }
\end{itemize}

Overall, working with \ac{OSS} was a positive experience in this project.
We believe that the advantages of working with open source outweigh the potential drawbacks.
This is especially true for contributors who are aware of the risks, and make efforts to mitigate them.

\section{Alternative Solutions}\label{sec:alternative_solutions}
We have learned from porting Peer-to-Peer Matrix to Android, and in the process we have come up with improvements and alternative methods which we discuss here.

\subsection{A Combined Client-Server}\label{subsec:combined_client_server}
While our approach to combine two Matrix components into a single app works, it is wasteful.
Although the Client and Server run in the same process, they communicate via HTTPS\@.
It would be much more effective to write them both in the same programming language and share the data in memory instead.
An alternative approach could therefore be to combine the two into a single application which implements federation with other homeservers, but otherwise limits the Client-Server API to the local device.

\subsection{Using WebAssembly}\label{subsec:using_webassembly}
While we have been developing Peer-to-Peer Matrix in Go and Java, the Matrix team has continued development of the WebAssembly implementation.
Since Android browsers are capable of using that solution as well, one might consider packaging Riot for Web and the WebAssembly Dendrite as an Android application.
While this solution would likely be less performant than one written for native Android, it would be better aligned with the priorities of the Matrix team.

\subsection{Fog Computing}\label{sec:fog_computing}
In a paper from 2015, Hassan et al suggest the use of fog computing to improve performance for applications which typically use the cloud\cite{hassan2015help}.
Fog computing is essentially introducing a new layer of computation between the mobile device and the cloud.
In practical terms, a single device at home could serve as a homeserver for multiple mobile devices.
This approach may be a practical compromise between cloud hosting a Matrix server and embedding it in a mobile device.
It allows users to manage their personal data at home, but avoids the excess battery and storage costs on the mobile device.

However, this approach may require hardware and software maintenance, and good security practices, so that the fog computing device does not become a weakness to be exploited by hackers or malware.
This extra complexity is easy to avoid with cloud offerings, where providers can offer a whole package deal.
By embedding the server in the mobile device, it can also be programmed for a simpler setup.
These arguments are summarized in Table~\ref{tab:alternatives}.


\begin{table}[]
	\centering
	\begin{tabular}{@{}lccc@{}}
		\toprule
		                 & \textbf{Cloud Offering} & \textbf{Fog Computing} & \textbf{Embedded Server} \\ \midrule
		Hardware Access  & No                      & Yes                    & Yes                      \\
		Subscription     & Yes                     & No                     & No                       \\
		Peer-to-Peer     & No\footnotemark         & Yes                    & Yes                      \\
		Space Efficient  & Yes                     & Yes                    & No                       \\
		Energy Efficient & Yes                     & Yes                    & No                       \\ \bottomrule
	\end{tabular}
	\caption{An overview of the differences between cloud computing, fog computing and embedded servers.}%
	\label{tab:alternatives}
\end{table}

\footnotetext{
	Cloud offerings do not offer Peer-to-Peer Matrix servers yet.
	Considering our contributions, and the current experiments that the Matrix foundation is conducting, this is likely to change.
}