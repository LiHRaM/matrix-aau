\chapter{Implementation}
Section 2.3.4 lists the benefits of Dendrite over Synapse for mobile clients.
In order to get Dendrite working on mobile, we need to merge the \github{matrix-org/dendrite/p2p} branch with the master branch, which supports embedded databases.
Android has excellent support for SQLite in apps, which is the data backing the Dendrite installation, and stores data such as account information and message metadata.

\section{Embedding Dendrite}
We begin by preparing Dendrite for embedding.
The \texttt{p2p} branch already supports \ac{P2P} messaging, but lacks support for embedded database technology.
We merge this branch with master and refactor it for use within Android by creating a \texttt{server} Go package.

\subsection{Dendrite Commands}
The commands available in the Dendrite repository are listed in listing \ref{fig:dendrite-cmd-tree}.
These commands are all located in the \texttt{cmd} subdirectory.


\begin{lstfloat}
  \begin{lstlisting}[
  numbers=none,
  xleftmargin=0pt,
  framexleftmargin=0pt,
  caption={
    Running \texttt{tree -L 1} from the \texttt{cmd} directory in the Dendrite repository.
    Each folder contains code for a separate command.
  },
  label={fig:dendrite-cmd-tree},
]
  dendrite/cmd
  ├─ client-api-proxy
  ├─ create-account
  ├─ create-room-events
  ├─ dendrite-appservice-server
  ├─ dendrite-client-api-server
  ├─ dendrite-demo-libp2p
  ├─ dendrite-edu-server
  ├─ dendrite-federation-api-server
  ├─ dendrite-federation-sender-server
  ├─ dendritejs
  ├─ dendrite-media-api-server
  ├─ dendrite-monolith-server
  ├─ dendrite-public-rooms-api-server
  ├─ dendrite-room-server
  ├─ dendrite-sync-api-server
  ├─ federation-api-proxy
  ├─ generate-keys
  ├─ kafka-producer
  ├─ mediaapi-integration-tests
  ├─ roomserver-integration-tests
  └─ syncserver-integration-tests
\end{lstlisting}
\end{lstfloat}

There are three kinds of commands:
\begin{itemize}
  \item{
        \textbf{Monolith}:
        The monolith is an independent instance of Dendrite, with all necessary services.
        \texttt{dendrite-monolith-server}, \texttt{dendritejs} and \texttt{dendrite-demo-libp2p} are examples of a monolith.
        }
  \item{
        \textbf{Service}:
        The service is a single component of the Dendrite homeserver, separated to make scaling at large more straightforward.
        }
  \item{
        \textbf{Test}:
        The test command runs integration tests for a specific component or group of components of Dendrite.
        }
\end{itemize}


\subsection{Peer-to-Peer Dendrite}
The work in \github{matrix-org/dendrite/p2p} at commit \texttt{af98bca}\footnote{
  As this branch has been merged into master, it may not exist at the time of reading.
  It can however be found at my fork at \github{LiHRaM/dendrite/p2p-historical}.
}
contains experiments which modify the \texttt{dendrite-monolith-server} to run as a \ac{P2P} server.
Key libraries are modified to account for the optional \ac{P2P} capabilities.
We merge these changes with the master branch in \github{matrix-org/dendrite\#956}.
These changes introduce a new command in the \texttt{cmd} folder: \texttt{dendrite-demo-libp2p}, and revert the changes which pollute the pre-existing non-p2p dendrite code.
Further refactoring is done to separate the command from the library\footnote{See these changes at \github{LiHRaM/dendrite/master.}}.
This separation allows us to generate Android bindings from the underlying library as well as running the command directly.

\subsection{Android Library}
The Go library has a single entrypoint as seen in listing \ref{lst:dendrite_init}.
From the Go library, the Gomobile command generates an Android Archive (.aar) file which can be referenced via Android Studio.
\begin{lstfloat}
  \begin{lstlisting}[
  language=Go, 
  label={lst:dendrite_init}, 
  caption={
    The Init function spawns a Dendrite server as a monolith.
    The monolith creates SQLite databases in the directory specified by \texttt{path}.
    The other two parameters are used to avoid namespace and port clashing.
    This is necessary for spawning more than one instance on the same host.
}]
// server.go
func Init(
    path string, 
    instanceName string, 
    instancePort int) {}
\end{lstlisting}
\end{lstfloat}

\subsection{Changing Riot and RiotX}
\begin{lstfloat}
  \begin{lstlisting}[language=Kotlin, label={lst:riotx_init}, caption={
  A simple approach to embedding Dendrite in RiotX is simply to spawn a new thread with a call to the \texttt{Init} method as shown in listing \ref{lst:dendrite_init}
}]
thread(start = true) {
    Dendrite.init(
        filesDir.path,
        "dendrite-server", 
        8080)
}
\end{lstlisting}
\end{lstfloat}
