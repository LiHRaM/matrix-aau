\chapter{Implementation}


% \ac{P2P} Matrix is already being actively developed in the Dendrite repository at \github{matrix-org/dendrite}.
% The work there is mostly focused on the implementation in JavaScript/WASM.
% The \texttt{p2p} branch contains progress on a pure Go implementation of \ac{P2P}.
% This branch does not support embedded databases though, whereas the master branch does.
% The first step toward a mobile implementation is therefore integrating the two branches.

\section{Dendrite P2P Integration}
The work in the \texttt{p2p} branch at commit \texttt{af98bca}
\footnote{
    As this branch has been merged into master, it may not exist at the time of reading. It can however be found at my fork at \github{LiHRaM/dendrite/p2p-historical}.
}
contains experiments which modified the \texttt{dendrite-monolith-server} to run as a \ac{P2P} server.
Key libraries have also been modified to account for the optional \ac{P2P} capabilities.
We merge these changes with the master branch in \github{matrix-org/dendrite\#956}.
These changes introduce a new command in the \texttt{cmd} folder: \texttt{dendrite-demo-libp2p}, and revert the changes which pollute the pre-existing non-p2p dendrite code.
\ac{P2P} libraries are extracted into libraries with the command.

Additionally, after the merge request, further refactoring is done to separate the command entirely from the library.
This separation allows us to run a P2P server on the computer to test against the embedded Dendrite on mobile.

\section{Embedding Dendrite}
Dendrite is embedded using Gomobile.
The library has a single entrypoint as seen in listing \ref{lst:dendrite_init}.
\begin{lstlisting}[language=Go, label={lst:dendrite_init}, caption={
    The Init function spawns a Dendrite server as a monolith.
    The monolith creates SQLite databases in the directory specified by \texttt{path}.
    The other two parameters are used to avoid namespace and port clashing.
    This is necessary for spawning more than one instance on the same host.
}]
func Init(
    path string, 
    instanceName string, 
    instancePort int) {}
\end{lstlisting}

\subsection{Embedding in Riot and RiotX}


\begin{lstlisting}[language=Kotlin, label={lst:riotx_init}, caption={
    A simple approach to embedding Dendrite in RiotX is simply to spawn a new thread with a call to the \texttt{Init} method as shown in listing \ref{lst:dendrite_init}
}]
thread(start = true) {
    Dendrite.init(
        filesDir.path,
        "dendrite-server", 
        8080)
}
\end{lstlisting}
