\chapter{Porting to Android}\label{chp:implementation}
This chapter documents the work done to implement the design illustrated in Figure~\ref{fig:design_overview}.
We begin by preparing Dendrite to run on Android in peer-to-peer mode.
Afterwards, we configure Riot to compile Dendrite as a part of the project.
Finally, we modify Riot to run Dendrite as it starts.

\section{Preparing Dendrite for Android}\label{sec:preparing_dendrite}
The first task is to prepare Dendrite for Android by adding support in the peer-to-peer version for embedded databases.
Embedded databases are a very convenient method for storing persistent data in Android, where the preferred method is to use the app's file directory for persistent data\cite{android_devdocs_data_storage, android_devdocs_internal_storage}, because applications are isolated from the rest of the system.

As shown in Table~\ref{tab:homeserver_comparison}, Dendrite does support embedded databases, but on a different version.
We approach this task by collaborating with the authors of the Dendrite project on merging the work which supports embedded databases with the work which supports peer-to-peer networking.
The work is documented in \github{matrix-org/dendrite/956}, and consists of the following steps:
\begin{enumerate}
	\item{
	      \textbf{Extracting Peer-to-Peer Networking}:
	      The original peer-to-peer version was an experiment which injected peer-to-peer logic into the main Dendrite libraries.
	      Instead, the code was refactored so that the peer-to-peer logic was in its own library, and could be composed with the main libraries into a decentralized application.
	      }
	\item{
	      \textbf{Creating a Demo Program}:
	      The Dendrite code repository consists of a collection of libraries, and \textit{commands}.
	      The commands are example programs which use the libraries to start parts of Dendrite in different configurations.
	      We add a command, \texttt{dendrite-demo-libp2p}, which set up and configured Dendrite to run in peer-to-peer mode.
	      }
\end{enumerate}
In concrete terms, this work merges the \github{matrix-org/dendrite/p2p} branch with \texttt{matrix\-org/dendrite/master}.

\section{Compiling Dendrite for Android}
The next step is to make sure that Dendrite can run on Android.
Dendrite is written in the Go programming language, which supports a variety of operating systems, including Android and~iOS\@.
The Gomobile program described in Section~\ref{sec:gomobile} compiles binaries and libraries for Android.
It also generates Android code via Java, which enables native Android applications to reference the interface directly.
To make this convenient, we create a new project at \github{LiHRaM/server}, which is a restructured version of the \texttt{dendrite-demo-libp2p} command.
The main difference is that the code is cleanly separated into a single library, \texttt{server}, and an optional command in a subfolder for testing.
We illustrate the structure of this project in Figure~\ref{fig:dirtree_server}.
The \texttt{server} library exports a single function, \texttt{Init}, with the signature seen in Listing~\ref{lst:dendrite_init}.
Init handles the logic for setting up for and starting a Dendrite server.
As shown in Figure~\ref{fig:design_overview}, Riot and Dendrite must synchronize before Riot can finish setting up.
We implement this by providing a \texttt{Callback} interface, which can be implemented on the Java side, and passed to the \texttt{Init} function.
When the \texttt{Init} function has found a suitable port, it calls \texttt{SetPort} on the interface object, which allows the client to store the port for later use.
Listing~\ref{lst:vector_callback} shows the callback we implement on the Java side.

\begin{figure}
	\dirtree{%
		.1 server/.
		.2 server.go.
		.2 p2pdendrite.go.
		.2 mdnslistener.go.
		.2 build.gradle.
		.2 Makefile.
		.2 server.aar.
		.2 main/.
		.2 storage/.
	}
	\caption{%
		The \github{LiHRaM/server} directory tree, with selected files.
		This directory is a part of the Android project, which references it through \texttt{build.gradle}.
		The \texttt{Makefile} includes a script which generates \texttt{server.aar}, which is included by the build file.
	}%
	\label{fig:dirtree_server}
\end{figure}

\begin{lstfloat}
	\begin{lstlisting}[
  language=Go,
  label={lst:dendrite_init},
  caption={
    The Init function spawns a Dendrite server.
    The server creates SQLite databases in the directory specified by \texttt{path}.
    \texttt{instanceName} is used as a prefix for the database files that the server creates.
    \texttt{instancePort} can be used to specify a port.
    If it the port is set as \texttt{0}, the port is automatically determined at run time.
    \texttt{callback} is used to synchronize with the client.
}]
// Callback provides the the caller a way to respond to the port being set.
type Callback interface {
	SetPort(int)
}

// Init starts the Dendrite server in p2p mode
func Init(
    path string, 
    instanceName string, 
    instancePort int, 
    callback Callback) {/* not included */}
\end{lstlisting}
\end{lstfloat}


\begin{lstfloat}
	\begin{lstlisting}[
    language=Java,
    label={lst:vector_callback},
    caption={
        \texttt{PortActivatedCallback} is a Java class which implements the interface defined in the Go \texttt{server} project.
        The method \texttt{getPort} returns the port on which the Dendrite server runs.
        To make sure that it returns the correct port, we use a primitive spin lock, by looping until the inner \texttt{mDendritePort} is no longer \texttt{0}.
        If this function were to be called without passing the object to the \texttt{Init} method, it would block forever.
    }
]
private static class PortActivatedCallback implements Callback {
    private long mDendritePort = 0;

    long getPort() {
        this.blockOnPort();
        return mDendritePort;
    }

    private void blockOnPort() {
        while (mDendritePort == 0) {

        }

        Log.d(LOG_TAG, "Port accessed :" + mDendritePort);
    }

    @Override
    public void setPort(long l) {
        mDendritePort = l;
    }
}
\end{lstlisting}
\end{lstfloat}

Gomobile can be used to bind the underlying library, which has the interface as shown in Listing~\ref{lst:dendrite_init}.
This command generates an \textbf{A}ndroid \textbf{Ar}chive \texttt{(.aar)} file\cite{android_devdocs_aar}, which includes the compiled Go code and Java bindings.

\section{Running Dendrite on Android}
Finally, we prepare the Android side to connect with and run Dendrite.
When a user starts the app, there are two main scenarios:
\begin{enumerate}
	\item{
	      \textbf{Registration}:
	      The user has not registered yet, in which case the user is taken to the registration screen.
	      }
	\item{
	      \textbf{Conversations}:
	      If the app can find information about a logged in user, it goes straight to the main page, which shows the conversations of the user.
	      }
\end{enumerate}
In the code, each case is handled differently.
Typically, when a user registers, the user also chooses which server they register with.
This information is registered along with the credentials of the user and stored for later.
In our case, however, the server is located on the device itself, which changes every time it starts up.
This means that we must change the server address stored with the credentials whenever they are fetched.
Otherwise, the user would not be able to access the server again.

Whenever the app starts up, it runs the code stored in \texttt{VectorApp.java}.
We modify the \texttt{onCreate} method, so that it starts the Dendrite server.
When Dendrite is done initializing, it passes the latest port to the client by calling \texttt{setPort}, which is defined in Listing~\ref{lst:vector_callback}.
Specifically, we introduce our changes into the \texttt{onCreate} method.
In Listing~\ref{lst:vectorapp}, we show the changes made to this entry point.

\begin{lstfloat}
	\begin{lstlisting}[
        language=Java,
        label={lst:vectorapp},
        caption={
            VectorApp is the main entry point of the Android application.
            \texttt{onCreate} is called every time the app is started.
            It is \textit{not} called when the app is resumed.
            We store the Dendrite server \ac{URL} in the \texttt{VectorApp} class, and provide the \texttt{getDendriteUrl} get method for external access.
        }
    ]
// The modified VectorApp in VectorApp.java
public class VectorApp extends MultiDexApplication {
    /* PortActivatedCallBack definition */

    private Uri mDendriteUrl;

    public Uri getDendriteUrl() {
        return mDendriteUrl;
    }

    @Override
    public void onCreate() {
        /* not included */
        PortActivatedCallback cb = new PortActivatedCallback();
        Runnable dendriteTask = () -> Server.init(
                getFilesDir().getPath(),
                "dendrite-server",
                0,
                cb
        );
        new Thread(dendriteTask).start();
        mDendriteUrl = Uri.parse("http://localhost:" + cb.getPort());
        /* not included */
    }
}
\end{lstlisting}
\end{lstfloat}

The registration process sets \url{https://matrix.org} as the default homeserver on registration.
We modify this to be the address of the newly created Dendrite server.
In Listing~\ref{lst:login_activity}, we rewrite the \texttt{getHomeServerUrl} to return the value stored by \texttt{VectorApp}.
Typically, when a user registers with a Matrix server, the \ac{URL} does not change.
The client therefore stores the homeserver \ac{URL} along with the user information.
We add logic which updates the homeserver \ac{URL} whenever the user information is fetched, as seen in Listing~\ref{lst:matrix_java}.

\begin{lstfloat}
	\begin{lstlisting}[
    language=Java,
    label={lst:login_activity},
    caption={
        \texttt{getHomeServerUrl} is called when the user attempts to register.
        We modify it to only returns the address of the embedded Dendrite server.
        A convenient consequence is that the code and registration process becomes simpler.
    }
]
// Before
private String getHomeServerUrl() {
    String url = ServerUrlsRepository.INSTANCE.getDefaultHomeServerUrl(this);

    if (mUseCustomHomeServersCheckbox.isChecked()) {
        url = mHomeServerText.getText().toString().trim();

        if (url.endsWith("/")) {
            url = url.substring(0, url.length() - 1);
        }
    }

    return url;
}

// After
private String getHomeServerUrl() {
    return VectorApp.getInstance().getDendriteUrl().toString();
}
\end{lstlisting}
\end{lstfloat}

\begin{lstfloat}
	\begin{lstlisting}[
    language=Java,
    label={lst:matrix_java},
    caption={
        \texttt{Matrix.java} contains a class which controls access to the Matrix \ac{SDK} along with session management.
        \texttt{getDefaultSession} is called when a user has already registered with the server.
        We modify it to update the homeserver \ac{URL} to match the embedded Dendrite server address.
    }
]
public synchronized MXSession getDefaultSession() {
    /* not included */
    for (HomeServerConnectionConfig config : hsConfigList) {

        // Update the port of the homeserver
        config.setHomeserverUri(
            VectorApp.getInstance().getDendriteUrl());

        /* not included */
    }
}
\end{lstlisting}
\end{lstfloat}

% \begin{lstfloat}
%     \begin{lstlisting}[
%     language=Java,
%     label={lst:label},
%     caption={
%         Lorem ipsum.
%     }
% ]
% // Code...
% \end{lstlisting}
% \end{lstfloat}

