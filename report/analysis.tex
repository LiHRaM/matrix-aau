\chapter{Analysis}
We begin the analysis by examining the Matrix Open Standard.

\section{The Matrix Standard}
The Matrix Standard defines several \ac{API}s.
Each of these \ac{API}s is expected to communicate with \ac{JSON} over \ac{REST}.
To be more precise, the following \ac{API}s are defined:
\begin{itemize}
    \item \textbf{Client-Server}: A \textit{homeserver} interacting with a \textit{client}. In peer-to-peer Matrix, the \textit{client} and \textit{homeserver} are both on-device.
    \item \textbf{Server-Server}: Two or more \textit{homeservers} interacting. The servers synchronize state via \textit{federation}.
    \item \textbf{Application Service}: Interoperability and extensibility is largely achieved through application services, which can \textit{bridge} external chat protocols to Matrix, allowing e.g. interoperability between \ac{IRC} and Matrix.
    \item \textbf{Identity Service}: Identity services enable external identities, such as email, mobile numbers, or other services to be attached to a Matrix user.
    \item \textbf{Push Gateway}: The way push notifications are implemented for Matrix events.
\end{itemize}

Each homeserver needs to be registered with a DNS entry, by which they are currently identified.
A user is registered under a homeserver, and identified thereby, e.g. \texttt{@user:homeserver.domain}.
The specification is meant to change, i.e. it was not designed as final, and it has an established process for changes.
This process is defined in \cite{matrix_org_spec_changes}.
In order to account for a changing standard, a versioning system was established by the Matrix foundation.
This standard is found in \cite{matrix_org_spec}, and establishes the foundation for interoperability between clients and chat rooms of varying versions.

\subsection{Client-Server}
The \textit{homeserver} is essentially a database with an \ac{API} which can be accessed by the client via \ac{HTTP} \ac{API} calls.
The client \ac{API} provides commands for sending messages, controlling rooms and synchronizing conversation history.
Clients may use both \ac{HTTP} and \ac{HTTPS}.

\subsection{Server-Server}
Due to the decentralized nature of the Matrix network, the server-server \ac{API} provides commands for synchronizing message history by pushing and retrieving messages, and sharing user information.
Servers communicate in three main ways:
\begin{itemize}
    \item \ac{PDU}s: Broadcast events which are historically important and shall be stored.
    \item \ac{EDU}s: 1 to 1 events which aren't historically important, don't need to be stored, and don't require a response.
    \item Queries: Requests for information and the corresponding response.
\end{itemize}
Servers have a strong requirement of using \ac{HTTPS}.

\subsection{Application Service}
Homeservers are extended via Application Service \ac{API}s, which are privileged plugins.
These plugins are meant to work on any \textit{homeserver} which correctly implements the Matrix standard.
Application Services observe events from homeservers, and can broadcast events into rooms in which they participate.
Homeservers are responsible for passing events to Application Services, which they can be configured to do via configuration files.

\subsection{Identity Service}
This \ac{API} establishes a method for mapping third-party identifiers to matrix identifiers.
This allows people to find Matrix users by alternative service identifiers, such as email or their telephone numbers, or accounts at different services.

\subsection{Push Gateway}
An API for pushing notifications to users when they arrive at the homeserver.

\section{Homeservers}
\subsection{Synapse}
\subsection{Dendrite}

\section{Mobile Clients}
\subsection{Riot}
\subsection{RiotX}

\section{Problem Statement}
\begin{itemize}
    \item TODO
\end{itemize}
