\section{Analysis}
We explore the different questions posed in the initial problem statement in section \ref{subsec:initial_problem_statement}.

\paragraph{The Matrix Standard} The Matrix standard defines the behavior of a \textit{homeserver}, and a \textit{client}\cite{matrix_org_spec}.
Each homeserver needs to be registered with a DNS entry, by which they are currently identified.
A user is registered under a homeserver, and identified thereby, e.g. \texttt{@user:homeserver.domain}.
The specification is meant to change, i.e. it was not designed as final, and it has an established process for change.
This process is defined in \cite{matrix_org_spec_changes}.
In order to account for a changing standard, a versioning system was established by the Matrix foundation.
This standard is found in \cite{matrix_org_spec}, and establishes the foundation for interoperability between clients and chat rooms of varying versions.

\subsection{Use of the Matrix Standard}
Matrix.org provide a free homeserver, from which the foundation reported the following statistics\cite{fosdem_event_p2p_matrix}:
\begin{itemize}
    \item \textasciitilde{}13.5 million users are visible from the Matrix.org homeserver.
    \item \textasciitilde{}5 million messages are sent each day.
    \item France, Germany and the US are reported to use Matrix, potentially also the UK.
    \item \textasciitilde{}100 companies reported using Matrix.
\end{itemize}

\subsection{Matrix Specification Limitations}
We consider some of the limitations of the Matrix Specification.

As described in Github issue \cite{github_matrix_170}, a fundamental limitation of the client-server architecture is that users require a separate homeserver, with a static IP address and a DNS entry.
To clarify, homeservers do not support changing domain names, as it is an essential part of their identity.
Not all users have the hardware or resources to host their own server, and the specification currently does not support peer-to-peer communication.

\subsection{Matrix Reference Implementation Limitations}
\subsection{Properties of Solutions}

\subsection{Problem Statement}
\begin{itemize}
    \item Which technical challenges must be overcome in order to implement peer-to-peer Matrix?
    \item 
    \item How could one implement such a solution?
\end{itemize}
