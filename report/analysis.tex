\chapter{Matrix and the Mobile World}
The initial problem statement has several elements, which we deconstruct into the following list:
\begin{itemize}
	\item{
	      \textbf{Peer-to-Peer Matrix}:
	      What is the current strategy of Matrix?
	      How is it different from peer-to-peer networking?
	      Are they mutually exclusive, or can they supplement each other?
	      }
	\item{
	      \textbf{Matrix on the Smartphone}:
	      How does Matrix currently work on mobiles?
	      What effect will introducing peer-to-peer networking have?
	      }
	\item{
	      \textbf{Mobile Constraints}:
	      What mobile constraints do we need to respect?
	      How do we evaluate the effect of our solution on these constraints?
	      }
\end{itemize}

By exploring the answers to these questions, we inform a more concrete problem statement.

\section{Peer-to-Peer Matrix}
To understand how Matrix works, we turn to the Matrix Standard, which defines what is currently possible.
Afterwards, we consider how it can be extended to include peer-to-peer communication.

\subsection{The Matrix Standard}
The Matrix standard defines how entities interact in the Matrix network.
An \ac{API} is a way in which programs can define how they communicate.
The Matrix Standard defines several \ac{API}s.

Each of these \ac{API}s is expected to communicate using \ac{JSON} over \ac{REST}.
\begin{itemize}
	\item \textbf{Client-Server}:
	      A \textit{homeserver} interacting with a \textit{client}.
	      In \ac{P2P} Matrix, the \textit{client} and \textit{homeserver} are both on-device.
	\item \textbf{Server-Server}:
	      Two or more \textit{homeservers} interacting.
	      The servers synchronize state via \textit{federation}.
	\item \textbf{Application Service}:
	      Interoperability and extensibility is largely achieved through application services, which can \textit{bridge} external chat protocols to Matrix, allowing e.g.~interoperability between \ac{IRC} and Matrix.
	\item \textbf{Identity Service}:
	      Identity services enable external identities, such as email, mobile numbers, or other services to be attached to a Matrix user.
	\item \textbf{Push Gateway}:
	      The way push notifications are implemented for Matrix events.
\end{itemize}

Each homeserver needs to be registered with a \ac{DNS} entry, by which they are identified.

\subsubsection{Central Concepts}
The standard can be decomposed into a set of entities which interact.
It is vital to understand how these entities interact, and how these interactions may affect a \ac{P2P} implementation.

\paragraph{User}
A user is registered with a homeserver, and identified thereby, e.g.~\texttt{@user:example.com}.
The user participates in conversations via rooms.
The user's chat history is managed by all homeservers with users which participate in these rooms.

\paragraph{Device}
A client application corresponds roughly to what is termed a device in the Matrix standard.
There are some caveats, for example, the same client on different browsers will be considered different devices per browser.
A smartphone with separate mobile applications will also be considered multiple devices per application.
The access of devices can be revoked via the Matrix \ac{API}, and they are used for managing encryption keys.
Clients register themselves as devices on the first login.

\paragraph{Event}
All state changes are represented by events.
Each event has a \texttt{type} field, which follows the Java package naming convention to avoid conflicts.

For example, a message event has the type~\texttt{m.room.message}.
The Matrix Standard reserves all types starting with the~\texttt{m.} prefix.
Events typically happen in the context of a room.

\paragraph{Event Graph}
Because of network partitions, we cannot guarantee that all homeservers are connected to each other at all times.
The specification specifies how events are structured, so that each homeserver may eventually become consistent with the others.
The events are stored in a partially ordered \ac{DAG}.
Events are assigned a monotonically increasing \texttt{depth} to ease comparison.

\paragraph{Room}
Rooms serve as the unit to which users can publish events.
They are a flexible concept, but also a simple one.
In order to send a direct message to another user, the user will have to create a room, invite the other user and publish a message to the room.
Rooms allow an arbitrary amount of participants, and can be managed by users with the sufficient authority.
Authority in rooms is represented by a power level, which is between 0 and 100.
The default power level is 0.
When users publish events to a room, only those with a sufficient power level receive the event.
Rooms have a single identifier, which is written like so: \texttt{!id:example.com}, and can have several aliases, which are written: \texttt{\#alias:domain}.
Although a room is identified by a domain, it is federated and therefore not hosted only on its original homeserver.
Rooms all have a specific version, which specifies how they operate.
When new features are implemented for rooms, they are made available by adding a new version.
Rooms change versions via upgrades.

\paragraph{Identity}
The standard supports mappings of third party identities to Matrix identities.
Examples of third party entities are email, social media accounts and phone numbers.

\paragraph{Profiles}
Users can store information in their profiles, such as images, display names, etc.

\subsubsection{Client-Server}
The \textit{homeserver} is essentially a database with an \ac{API} which can be accessed by the client via \ac{HTTP} \ac{API} calls.
The client \ac{API} provides commands for sending messages, controlling rooms and synchronizing conversation history.

Clients may use both \ac{HTTP} and \ac{HTTPS}.

\subsubsection{Server-Server}
Due to the decentralized nature of the Matrix network, the server-server \ac{API} provides commands for synchronizing message history by pushing and retrieving messages, and sharing user information.
Servers communicate in three main ways:
\begin{itemize}
	\item{
	      \textbf{\ac{PDU}:}
	      A broadcast event which is historically important and shall be stored.
	      \textit{Example}: A message was sent.
	      }
	\item{
	      \textbf{\ac{EDU}:}
	      A 1-to-1 event which is not historically important, does not need to be stored, and does not require a response.
	      \textit{Example}: A user is typing a message.
	      }
	\item{
	      \textbf{Queries:}
	      Requests for information and the corresponding response.
	      }
\end{itemize}
Servers must use \ac{HTTPS}, although it is possible to go without, for example while debugging or testing locally.

\subsubsection{Application Service}
Homeservers are extended via Application Service \ac{API}s, which are privileged plugins.
These plugins are meant to work on any \textit{homeserver} which correctly implements the Matrix standard.
Application Services observe events from homeservers, and can broadcast events into rooms in which they participate.
Homeservers are responsible for passing events to Application Services, which they can be configured to do via configuration files.

\subsubsection{Identity Service}
This \ac{API} establishes a method for mapping third-party identifiers to Matrix identifiers.
This allows people to find Matrix users by alternative service identifiers, such as email or their telephone numbers, or accounts at different services.

\subsubsection{Push Gateway}
The homeserver can forward events to the Push Gateway, which then passes this information to push notification services.
While a dedicated homeserver can be expected to run at all times, a \ac{P2P} homeserver running on a mobile device should only run when strictly necessary.
Push notifications could therefore be useful, but would have to be external to the homeserver running on-device.


\subsection{Matrix Spec Changes}
Although the Matrix specification at this time may not include peer-to-peer networking, that does not mean it's not a planned feature.
There is a defined process for proposing changes to the specification, which is known as Matrix Spec Changes\cite{matrix_org_spec_changes}.
This process allows third parties to contribute to the Matrix standard and provide assistance with open proposals.
With regard to \ac{P2P} Matrix on mobile devices, there are already open proposals:
\begin{itemize}
	\item{
	      \textbf{Peer-to-Peer Matrix}:
	      \github{matrix-org/matrix-doc\#710} is a proposal for implementing \ac{P2P} Matrix.
	      The Matrix team announced that they were already working on a solution, which is available for testing at \url{p2p.riot.im}.
	      This solution leverages internet technologies to run a homeserver in the browser with the client.
	      }
	\item{
	      \textbf{Centralized Accounts}:
	      Although Matrix is a decentralized protocol, accounts are currently centralized.
	      As a \ac{P2P} implementation would be running a homeserver per device, a user would have to register multiple accounts.
	      There is already an issue for decentralized user accounts on Matrix in \github{matrix-org/matrix-doc\#915}.
	      }
	\item{
	      \textbf{Homeserver Identities}:
	      A homeserver is identified by a DNS record, which makes it impossible to roam between networks and IP addresses on mobile devices.
	      This issue has been raised several times, for example in \github{matrix-org/matrix-doc\#710,712}.
	      }
\end{itemize}

From these proposals, it is clear that supporting peer-to-peer networking is on the list of priorities for Matrix.

\section{Matrix on the Smartphone}\label{sec:matrix_on_the_smartphone}
Introducing Peer-to-Peer Matrix does not necessarily have to have any effect on the current Matrix clients for Android.
That's because the Matrix Standard treats Client-Server communication and Server-Server communication as separate.
Mobile clients are only affected by changes in the Client-Server API, whereas the Peer-to-Peer changes occur in the Server-Server API\@.
That said, it might be beneficial to make changes to the mobile clients.
We consider the different available homeservers and clients to understand their strengths and differences.

\subsection{Homeservers}\label{sec:official_homeservers}
The Matrix organization has two official homeservers, Synapse\cite{matrix_org_synapse} and Dendrite\cite{matrix_org_dendrite}.

Homeservers perform the bulk of the work in the Matrix network.
They are the only component in the Matrix standard that needs to change in order to implement \ac{P2P} networking\cite{fosdem_event_p2p_matrix}.

\subsubsection{Synapse}
Synapse is the reference implementation of the homeserver \ac{API}s.
It is considered stable and feature-complete with regard to the specification.
It is hosted at \github{matrix-org/synapse}.

\paragraph{Synapse Code}
Synapse is written in Python 3\footnote{As explained by the installation guide: \url{https://github.com/matrix-org/synapse/blob/master/INSTALL.md}}.
Python is an interpreted language, which has negative performance implications.
However, some of the backing libraries of Synapse are implemented using C, which does not require the Python interpreter.
Synapse is packaged as a Python module, and can be installed and run via Python.

\subsubsection{Dendrite}\label{sec:dendrite}
Because of the performance and scalability issues that Synapse faces, the Matrix team started the development of a next-generation homeserver known as Dendrite, written in Go.
Peer-to-peer networking is provided by libp2p, which was chosen due to preëxisting implementations in Go and JavaScript.
Dendrite is hosted at \github{matrix-org/dendrite}.

\begin{figure}
	\centering
	\resizebox{!}{!}{% Graphic for TeX using PGF
% Title: /home/lihram/git/github.com/lihram/p8-matrix-p2p/report/graphics/dendrite_design.dia
% Creator: Dia v0.97.3
% CreationDate: Thu Apr  2 15:50:44 2020
% For: lihram
% \usepackage{tikz}
% The following commands are not supported in PSTricks at present
% We define them conditionally, so when they are implemented,
% this pgf file will use them.
\ifx\du\undefined
  \newlength{\du}
\fi
\setlength{\du}{15\unitlength}
\begin{tikzpicture}
\pgftransformxscale{1.000000}
\pgftransformyscale{-1.000000}
\definecolor{dialinecolor}{rgb}{0.000000, 0.000000, 0.000000}
\pgfsetstrokecolor{dialinecolor}
\definecolor{dialinecolor}{rgb}{1.000000, 1.000000, 1.000000}
\pgfsetfillcolor{dialinecolor}
% setfont left to latex
\definecolor{dialinecolor}{rgb}{0.000000, 0.000000, 0.000000}
\pgfsetstrokecolor{dialinecolor}
\node[anchor=west] at (-16.000000\du,7.000000\du){Event Logs};
\pgfsetlinewidth{0.200000\du}
\pgfsetdash{{\pgflinewidth}{0.200000\du}}{0cm}
\pgfsetdash{{\pgflinewidth}{0.200000\du}}{0cm}
\pgfsetmiterjoin
\pgfsetbuttcap
{
\definecolor{dialinecolor}{rgb}{0.000000, 0.000000, 0.000000}
\pgfsetfillcolor{dialinecolor}
% was here!!!
\pgfsetarrowsend{stealth}
{\pgfsetcornersarced{\pgfpoint{0.000000\du}{0.000000\du}}\definecolor{dialinecolor}{rgb}{0.000000, 0.000000, 0.000000}
\pgfsetstrokecolor{dialinecolor}
\draw (-10.000000\du,6.000000\du)--(-10.000000\du,7.500000\du)--(-19.000000\du,7.500000\du)--(-19.000000\du,9.000000\du);
}}
\definecolor{dialinecolor}{rgb}{1.000000, 1.000000, 1.000000}
\pgfsetfillcolor{dialinecolor}
\fill (-21.000000\du,11.000000\du)--(-21.000000\du,14.550000\du)--(-15.380000\du,14.550000\du)--(-15.380000\du,11.000000\du)--cycle;
\pgfsetlinewidth{0.100000\du}
\pgfsetdash{}{0pt}
\pgfsetdash{}{0pt}
\pgfsetmiterjoin
\definecolor{dialinecolor}{rgb}{0.000000, 0.000000, 0.000000}
\pgfsetstrokecolor{dialinecolor}
\draw (-21.000000\du,11.000000\du)--(-21.000000\du,14.550000\du)--(-15.380000\du,14.550000\du)--(-15.380000\du,11.000000\du)--cycle;
% setfont left to latex
\definecolor{dialinecolor}{rgb}{0.000000, 0.000000, 0.000000}
\pgfsetstrokecolor{dialinecolor}
\node at (-18.190000\du,12.969053\du){Reader};
\definecolor{dialinecolor}{rgb}{1.000000, 1.000000, 1.000000}
\pgfsetfillcolor{dialinecolor}
\fill (-21.460000\du,10.475000\du)--(-21.460000\du,14.025000\du)--(-15.840000\du,14.025000\du)--(-15.840000\du,10.475000\du)--cycle;
\pgfsetlinewidth{0.100000\du}
\pgfsetdash{}{0pt}
\pgfsetdash{}{0pt}
\pgfsetmiterjoin
\definecolor{dialinecolor}{rgb}{0.000000, 0.000000, 0.000000}
\pgfsetstrokecolor{dialinecolor}
\draw (-21.460000\du,10.475000\du)--(-21.460000\du,14.025000\du)--(-15.840000\du,14.025000\du)--(-15.840000\du,10.475000\du)--cycle;
% setfont left to latex
\definecolor{dialinecolor}{rgb}{0.000000, 0.000000, 0.000000}
\pgfsetstrokecolor{dialinecolor}
\node at (-18.650000\du,12.444053\du){Reader};
\definecolor{dialinecolor}{rgb}{1.000000, 1.000000, 1.000000}
\pgfsetfillcolor{dialinecolor}
\fill (-22.047500\du,9.912500\du)--(-22.047500\du,13.462500\du)--(-16.427500\du,13.462500\du)--(-16.427500\du,9.912500\du)--cycle;
\pgfsetlinewidth{0.100000\du}
\pgfsetdash{}{0pt}
\pgfsetdash{}{0pt}
\pgfsetmiterjoin
\definecolor{dialinecolor}{rgb}{0.000000, 0.000000, 0.000000}
\pgfsetstrokecolor{dialinecolor}
\draw (-22.047500\du,9.912500\du)--(-22.047500\du,13.462500\du)--(-16.427500\du,13.462500\du)--(-16.427500\du,9.912500\du)--cycle;
% setfont left to latex
\definecolor{dialinecolor}{rgb}{0.000000, 0.000000, 0.000000}
\pgfsetstrokecolor{dialinecolor}
\node at (-19.237500\du,11.881553\du){Readers};
\pgfsetlinewidth{0.100000\du}
\pgfsetdash{}{0pt}
\pgfsetdash{}{0pt}
\pgfsetbuttcap
{
\definecolor{dialinecolor}{rgb}{0.000000, 0.000000, 0.000000}
\pgfsetfillcolor{dialinecolor}
% was here!!!
\pgfsetarrowsend{latex}
\definecolor{dialinecolor}{rgb}{0.000000, 0.000000, 0.000000}
\pgfsetstrokecolor{dialinecolor}
\draw (-14.000000\du,11.000000\du)--(-7.850000\du,11.000000\du);
}
% setfont left to latex
\definecolor{dialinecolor}{rgb}{0.000000, 0.000000, 0.000000}
\pgfsetstrokecolor{dialinecolor}
\node[anchor=west] at (-14.400000\du,10.450000\du){Client-Server APIs};
\pgfsetlinewidth{0.100000\du}
\pgfsetdash{}{0pt}
\pgfsetdash{}{0pt}
\pgfsetbuttcap
{
\definecolor{dialinecolor}{rgb}{0.000000, 0.000000, 0.000000}
\pgfsetfillcolor{dialinecolor}
% was here!!!
\pgfsetarrowsend{latex}
\definecolor{dialinecolor}{rgb}{0.000000, 0.000000, 0.000000}
\pgfsetstrokecolor{dialinecolor}
\draw (-14.000000\du,14.000000\du)--(-7.850000\du,14.000000\du);
}
% setfont left to latex
\definecolor{dialinecolor}{rgb}{0.000000, 0.000000, 0.000000}
\pgfsetstrokecolor{dialinecolor}
\node[anchor=west] at (-14.400000\du,13.450000\du){Server-Server APIs};
\definecolor{dialinecolor}{rgb}{1.000000, 1.000000, 1.000000}
\pgfsetfillcolor{dialinecolor}
\fill (-12.635000\du,1.400000\du)--(-12.635000\du,4.950000\du)--(-7.015000\du,4.950000\du)--(-7.015000\du,1.400000\du)--cycle;
\pgfsetlinewidth{0.100000\du}
\pgfsetdash{}{0pt}
\pgfsetdash{}{0pt}
\pgfsetmiterjoin
\definecolor{dialinecolor}{rgb}{0.000000, 0.000000, 0.000000}
\pgfsetstrokecolor{dialinecolor}
\draw (-12.635000\du,1.400000\du)--(-12.635000\du,4.950000\du)--(-7.015000\du,4.950000\du)--(-7.015000\du,1.400000\du)--cycle;
% setfont left to latex
\definecolor{dialinecolor}{rgb}{0.000000, 0.000000, 0.000000}
\pgfsetstrokecolor{dialinecolor}
\node at (-9.825000\du,3.369053\du){Reader};
\definecolor{dialinecolor}{rgb}{1.000000, 1.000000, 1.000000}
\pgfsetfillcolor{dialinecolor}
\fill (-13.095000\du,0.875000\du)--(-13.095000\du,4.425000\du)--(-7.475000\du,4.425000\du)--(-7.475000\du,0.875000\du)--cycle;
\pgfsetlinewidth{0.100000\du}
\pgfsetdash{}{0pt}
\pgfsetdash{}{0pt}
\pgfsetmiterjoin
\definecolor{dialinecolor}{rgb}{0.000000, 0.000000, 0.000000}
\pgfsetstrokecolor{dialinecolor}
\draw (-13.095000\du,0.875000\du)--(-13.095000\du,4.425000\du)--(-7.475000\du,4.425000\du)--(-7.475000\du,0.875000\du)--cycle;
% setfont left to latex
\definecolor{dialinecolor}{rgb}{0.000000, 0.000000, 0.000000}
\pgfsetstrokecolor{dialinecolor}
\node at (-10.285000\du,2.844053\du){Reader};
\definecolor{dialinecolor}{rgb}{1.000000, 1.000000, 1.000000}
\pgfsetfillcolor{dialinecolor}
\fill (-13.682500\du,0.312500\du)--(-13.682500\du,3.862500\du)--(-8.062500\du,3.862500\du)--(-8.062500\du,0.312500\du)--cycle;
\pgfsetlinewidth{0.100000\du}
\pgfsetdash{}{0pt}
\pgfsetdash{}{0pt}
\pgfsetmiterjoin
\definecolor{dialinecolor}{rgb}{0.000000, 0.000000, 0.000000}
\pgfsetstrokecolor{dialinecolor}
\draw (-13.682500\du,0.312500\du)--(-13.682500\du,3.862500\du)--(-8.062500\du,3.862500\du)--(-8.062500\du,0.312500\du)--cycle;
% setfont left to latex
\definecolor{dialinecolor}{rgb}{0.000000, 0.000000, 0.000000}
\pgfsetstrokecolor{dialinecolor}
\node at (-10.872500\du,2.281553\du){Writers};
\pgfsetlinewidth{0.100000\du}
\pgfsetdash{}{0pt}
\pgfsetdash{}{0pt}
\pgfsetbuttcap
{
\definecolor{dialinecolor}{rgb}{0.000000, 0.000000, 0.000000}
\pgfsetfillcolor{dialinecolor}
% was here!!!
\pgfsetarrowsend{latex}
\definecolor{dialinecolor}{rgb}{0.000000, 0.000000, 0.000000}
\pgfsetstrokecolor{dialinecolor}
\draw (-20.937500\du,1.137500\du)--(-14.787500\du,1.137500\du);
}
% setfont left to latex
\definecolor{dialinecolor}{rgb}{0.000000, 0.000000, 0.000000}
\pgfsetstrokecolor{dialinecolor}
\node[anchor=west] at (-21.337500\du,0.587500\du){Client-Server APIs};
\pgfsetlinewidth{0.100000\du}
\pgfsetdash{}{0pt}
\pgfsetdash{}{0pt}
\pgfsetbuttcap
{
\definecolor{dialinecolor}{rgb}{0.000000, 0.000000, 0.000000}
\pgfsetfillcolor{dialinecolor}
% was here!!!
\pgfsetarrowsend{latex}
\definecolor{dialinecolor}{rgb}{0.000000, 0.000000, 0.000000}
\pgfsetstrokecolor{dialinecolor}
\draw (-20.982500\du,4.348501\du)--(-14.832500\du,4.348501\du);
}
% setfont left to latex
\definecolor{dialinecolor}{rgb}{0.000000, 0.000000, 0.000000}
\pgfsetstrokecolor{dialinecolor}
\node[anchor=west] at (-21.382500\du,3.798501\du){Server-Server APIs};
\end{tikzpicture}
}
	\caption{
		The homeserver can be separated into two main components: \textit{writers} and \textit{readers}.
		Writers respond to API requests by appending to the event log.
		Readers respond to API requests by returning information from the event log.\cite{dendrite_design_md}.
	}%
	\label{fig:dendrite_design}
\end{figure}

\paragraph{Dendrite Architecture}
The Dendrite design aims to increase simplicity and improve scalability issues in Synapse by using a log based architecture for events.
The central concept is therefore the \textit{event log} which is a history of everything that has happened in a room.
The event log is append only, and accessed by two types of components: \textit{writers} and \textit{readers}.
An example of a writer could be the API that marks a user as active.
A reader could be the logic that synchronizes state with the client.
Figure~\ref{fig:dendrite_design} provides a high-level illustration of the flow of data in this design.
Writers and readers must also have some way of computing the room state from the event log, or querying the room state from a central mediator component, such as a room server.
This allows writers to verify that an event is valid before appending it to the log.
Readers in a similar manner use the room state to know which servers and users have enough permissions to view the events.

\paragraph{Dendrite Code}
Dendrite is written in Go\cite{golang_org} and separated into several libraries, which can be accessed through commands.
The Go programming language supports Android compilation through Gomobile\cite{gomobile}, and Dendrite should therefore be portable to the smartphone.
The \ac{P2P} code is implemented on the \texttt{master} branch for JavaScript/Wasm, and depends on SQLite, which allows the homeserver to run completely in the browser.
Unfortunately, the JavaScript/WASM implementation of \ac{P2P} Dendrite does not support cross-compilation for Android, and must instead be in pure Go.
The \texttt{p2p} branch has a pure Go implementation of \ac{P2P}, but does not support SQLite.
In order to use the latest developments for \ac{P2P} Matrix, the \texttt{p2p} branch must be merged with \texttt{master}.

\paragraph{Libp2p}
Dendrite uses \ac{P2P} libraries provided by libp2p, which is a collection of protocols, libraries and implementations for peer-to-peer networking\cite{libp2p}.
The JavaScript implementation is capable of peer discovery across the internet using a protocol known as the \ac{WSS} protocol, the source code is at \github{libp2p/js-libp2p-websocket-star}.
Unfortunately, this protocol is not supported in Go\cite{libp2p_implementations}.
Instead, the Go implementation supports peer discovery on the local network using \ac{mDNS}.

\subsubsection{Gomobile}\label{sec:gomobile}
Gomobile is a Go program which extends the Go compiler by providing commands for generating Android and iOS apps and libraries\cite{gomobile}.
Generating an app with Gomobile requires that the entire application stack is written in Go, which includes the user interface and client logic.
This approach would require a client application to be written in Go.
Alternatively, Gomobile can generate Java bindings from Go libraries.
Riot and RiotX are written in Java and Kotlin respectively, and can therefore both reference the bindings generated by Gomobile from Go libraries.
The second approach would require less work, as existing code can be used for both the client and server.

\subsection{Mobile Clients}\label{sec:official_clients}
Matthew Hodgson, the co-founder of Matrix, explains in his talk on \ac{P2P} at FOSDEM\cite{fosdem_event_p2p_matrix} that client applications do not need to be modified in order to work with \ac{P2P} homeservers.
In fact, the clients and the homeservers communicate in exactly the same way.
It is therefore only the way that servers communicate with each other that changes.
This implies the possibility of embedding a homeserver into a mobile client.
Currently, Matrix has two official clients for Android, namely Riot and RiotX.
There are other open source alternatives which we do not consider in this report, as they are not officially supported and therefore not guaranteed to be maintained.
Additionally, we do not consider iOS, as the author of this report has no access to such a device.

\subsubsection{Riot}
Riot is the current stable client for Android phones.
It supports a large part of the Matrix standard, including older room versions and homeservers.
It is capable of registering an account and logging in on a Dendrite homeserver.
According to the introductory \texttt{README.md} on the official repository at \github{vector-im/riot-android}, development has mostly shifted to RiotX.

\subsubsection{RiotX}
RiotX is an experimental client at a similar stage as Dendrite, i.e.~not feature complete, but intended as the reference client for Android in the future.
It is mostly written in Kotlin, which is currently the recommended language by Google for Android development.
It is not capable of connecting to a Dendrite homeserver currently, complaining that it is an outdated homeserver.

\section{Mobile Constraints}\label{sec:mobile_constraints}
We begin with an overview of the mainstream constraints of using mobile devices.

\begin{itemize}
	\item{
	      \textbf{Battery Life}:
	      The most fundamental constraint of mobile devices is battery use.
	      The Android developer documentation describes battery life as ``the single most important aspect of the mobile user experience''\cite{android_devdocs_power}.
	      The reasoning is straightforward: if your mobile device has no power, it cannot be used.
	      They recommend the use of \textit{Battery Historian}\cite{battery_historian} to measure the battery use of applications.
	      }
	\item{
	      \textbf{Storage Space}:
	      A typical Android device might have somewhere between 16--128GB of memory available, with many older devices in the lower end of that range.
	      The Android developer documentation recommends trying to keep the memory use of applications low, especially in emerging markets\cite{android_devdocs_apksize}.
	      A Matrix homeserver might end up storing considerable amounts of data.
	      Although text formats are compact, the main impact of this metadata would be the storage of media, such as videos, pictures or sound files.
	      }
\end{itemize}

We want to evaluate the impact of our changes to preëxisting programs fairly.
To do this, we establish a baseline of the current battery life impact and storage space use, which we measure the modified program against.
As a part of the design, we may also set a goal for the impact that our modifications have on the baseline.

\begin{figure}
	\centering
	\resizebox{!}{!}{% Graphic for TeX using PGF
% Title: /home/lihram/git/github.com/lihram/matrix-aau/report/graphics/general-arch.dia
% Creator: Dia v0.97+git
% CreationDate: Thu May 14 17:51:50 2020
% For: lihram
% \usepackage{tikz}
% The following commands are not supported in PSTricks at present
% We define them conditionally, so when they are implemented,
% this pgf file will use them.
\ifx\du\undefined
  \newlength{\du}
\fi
\setlength{\du}{15\unitlength}
\begin{tikzpicture}[even odd rule]
\pgftransformxscale{1.000000}
\pgftransformyscale{-1.000000}
\definecolor{dialinecolor}{rgb}{0.000000, 0.000000, 0.000000}
\pgfsetstrokecolor{dialinecolor}
\pgfsetstrokeopacity{1.000000}
\definecolor{diafillcolor}{rgb}{1.000000, 1.000000, 1.000000}
\pgfsetfillcolor{diafillcolor}
\pgfsetfillopacity{1.000000}
\pgfsetlinewidth{0.100000\du}
\pgfsetdash{}{0pt}
\pgfsetbuttcap
\pgfsetmiterjoin
\pgfsetlinewidth{0.100000\du}
\pgfsetbuttcap
\pgfsetmiterjoin
\pgfsetdash{}{0pt}
{\pgfsetcornersarced{\pgfpoint{0.000000\du}{0.000000\du}}\definecolor{diafillcolor}{rgb}{1.000000, 1.000000, 1.000000}
\pgfsetfillcolor{diafillcolor}
\pgfsetfillopacity{1.000000}
\fill (-25.845103\du,2.621480\du)--(-25.845103\du,15.127314\du)--(-13.742683\du,15.127314\du)--(-13.742683\du,2.621480\du)--cycle;
}{\pgfsetcornersarced{\pgfpoint{0.000000\du}{0.000000\du}}\definecolor{dialinecolor}{rgb}{0.000000, 0.000000, 0.000000}
\pgfsetstrokecolor{dialinecolor}
\pgfsetstrokeopacity{1.000000}
\draw (-25.845103\du,2.621480\du)--(-25.845103\du,15.127314\du)--(-13.742683\du,15.127314\du)--(-13.742683\du,2.621480\du)--cycle;
}% setfont left to latex
\definecolor{dialinecolor}{rgb}{0.000000, 0.000000, 0.000000}
\pgfsetstrokecolor{dialinecolor}
\pgfsetstrokeopacity{1.000000}
\definecolor{diafillcolor}{rgb}{0.000000, 0.000000, 0.000000}
\pgfsetfillcolor{diafillcolor}
\pgfsetfillopacity{1.000000}
\node[anchor=base west,inner sep=0pt,outer sep=0pt,color=dialinecolor] at (-21.092110\du,4.507698\du){Client};
\pgfsetlinewidth{0.100000\du}
\pgfsetdash{}{0pt}
\pgfsetbuttcap
\pgfsetmiterjoin
\pgfsetlinewidth{0.100000\du}
\pgfsetbuttcap
\pgfsetmiterjoin
\pgfsetdash{}{0pt}
{\pgfsetcornersarced{\pgfpoint{0.000000\du}{0.000000\du}}\definecolor{diafillcolor}{rgb}{1.000000, 1.000000, 1.000000}
\pgfsetfillcolor{diafillcolor}
\pgfsetfillopacity{1.000000}
\fill (-22.571683\du,7.392252\du)--(-22.571683\du,12.838996\du)--(-17.300640\du,12.838996\du)--(-17.300640\du,7.392252\du)--cycle;
}{\pgfsetcornersarced{\pgfpoint{0.000000\du}{0.000000\du}}\definecolor{dialinecolor}{rgb}{0.000000, 0.000000, 0.000000}
\pgfsetstrokecolor{dialinecolor}
\pgfsetstrokeopacity{1.000000}
\draw (-22.571683\du,7.392252\du)--(-22.571683\du,12.838996\du)--(-17.300640\du,12.838996\du)--(-17.300640\du,7.392252\du)--cycle;
}% setfont left to latex
\definecolor{dialinecolor}{rgb}{0.000000, 0.000000, 0.000000}
\pgfsetstrokecolor{dialinecolor}
\pgfsetstrokeopacity{1.000000}
\definecolor{diafillcolor}{rgb}{0.000000, 0.000000, 0.000000}
\pgfsetfillcolor{diafillcolor}
\pgfsetfillopacity{1.000000}
\node[anchor=base west,inner sep=0pt,outer sep=0pt,color=dialinecolor] at (-21.138243\du,10.363111\du){Server};
\pgfsetlinewidth{0.100000\du}
\pgfsetdash{}{0pt}
\pgfsetbuttcap
{
\definecolor{diafillcolor}{rgb}{0.000000, 0.000000, 0.000000}
\pgfsetfillcolor{diafillcolor}
\pgfsetfillopacity{1.000000}
% was here!!!
\pgfsetarrowsend{stealth}
\definecolor{dialinecolor}{rgb}{0.000000, 0.000000, 0.000000}
\pgfsetstrokecolor{dialinecolor}
\pgfsetstrokeopacity{1.000000}
\draw (-20.425075\du,8.419093\du)--(-20.425075\du,5.691694\du);
}
\pgfsetlinewidth{0.100000\du}
\pgfsetdash{}{0pt}
\pgfsetbuttcap
{
\definecolor{diafillcolor}{rgb}{0.000000, 0.000000, 0.000000}
\pgfsetfillcolor{diafillcolor}
\pgfsetfillopacity{1.000000}
% was here!!!
\pgfsetarrowsstart{stealth}
\definecolor{dialinecolor}{rgb}{0.000000, 0.000000, 0.000000}
\pgfsetstrokecolor{dialinecolor}
\pgfsetstrokeopacity{1.000000}
\draw (-19.548997\du,8.454909\du)--(-19.548997\du,5.727510\du);
}
\pgfsetlinewidth{0.100000\du}
\pgfsetdash{}{0pt}
\pgfsetbuttcap
\pgfsetmiterjoin
\pgfsetlinewidth{0.100000\du}
\pgfsetbuttcap
\pgfsetmiterjoin
\pgfsetdash{}{0pt}
{\pgfsetcornersarced{\pgfpoint{0.000000\du}{0.000000\du}}\definecolor{diafillcolor}{rgb}{1.000000, 1.000000, 1.000000}
\pgfsetfillcolor{diafillcolor}
\pgfsetfillopacity{1.000000}
\fill (-10.201188\du,2.587974\du)--(-10.201188\du,15.093808\du)--(1.901231\du,15.093808\du)--(1.901231\du,2.587974\du)--cycle;
}{\pgfsetcornersarced{\pgfpoint{0.000000\du}{0.000000\du}}\definecolor{dialinecolor}{rgb}{0.000000, 0.000000, 0.000000}
\pgfsetstrokecolor{dialinecolor}
\pgfsetstrokeopacity{1.000000}
\draw (-10.201188\du,2.587974\du)--(-10.201188\du,15.093808\du)--(1.901231\du,15.093808\du)--(1.901231\du,2.587974\du)--cycle;
}% setfont left to latex
\definecolor{dialinecolor}{rgb}{0.000000, 0.000000, 0.000000}
\pgfsetstrokecolor{dialinecolor}
\pgfsetstrokeopacity{1.000000}
\definecolor{diafillcolor}{rgb}{0.000000, 0.000000, 0.000000}
\pgfsetfillcolor{diafillcolor}
\pgfsetfillopacity{1.000000}
\node[anchor=base west,inner sep=0pt,outer sep=0pt,color=dialinecolor] at (-5.448195\du,4.474192\du){Client};
\pgfsetlinewidth{0.100000\du}
\pgfsetdash{}{0pt}
\pgfsetbuttcap
\pgfsetmiterjoin
\pgfsetlinewidth{0.100000\du}
\pgfsetbuttcap
\pgfsetmiterjoin
\pgfsetdash{}{0pt}
{\pgfsetcornersarced{\pgfpoint{0.000000\du}{0.000000\du}}\definecolor{diafillcolor}{rgb}{1.000000, 1.000000, 1.000000}
\pgfsetfillcolor{diafillcolor}
\pgfsetfillopacity{1.000000}
\fill (-6.927768\du,7.358746\du)--(-6.927768\du,12.805490\du)--(-1.656725\du,12.805490\du)--(-1.656725\du,7.358746\du)--cycle;
}{\pgfsetcornersarced{\pgfpoint{0.000000\du}{0.000000\du}}\definecolor{dialinecolor}{rgb}{0.000000, 0.000000, 0.000000}
\pgfsetstrokecolor{dialinecolor}
\pgfsetstrokeopacity{1.000000}
\draw (-6.927768\du,7.358746\du)--(-6.927768\du,12.805490\du)--(-1.656725\du,12.805490\du)--(-1.656725\du,7.358746\du)--cycle;
}% setfont left to latex
\definecolor{dialinecolor}{rgb}{0.000000, 0.000000, 0.000000}
\pgfsetstrokecolor{dialinecolor}
\pgfsetstrokeopacity{1.000000}
\definecolor{diafillcolor}{rgb}{0.000000, 0.000000, 0.000000}
\pgfsetfillcolor{diafillcolor}
\pgfsetfillopacity{1.000000}
\node[anchor=base west,inner sep=0pt,outer sep=0pt,color=dialinecolor] at (-5.494328\du,10.329605\du){Server};
\pgfsetlinewidth{0.100000\du}
\pgfsetdash{}{0pt}
\pgfsetbuttcap
{
\definecolor{diafillcolor}{rgb}{0.000000, 0.000000, 0.000000}
\pgfsetfillcolor{diafillcolor}
\pgfsetfillopacity{1.000000}
% was here!!!
\pgfsetarrowsend{stealth}
\definecolor{dialinecolor}{rgb}{0.000000, 0.000000, 0.000000}
\pgfsetstrokecolor{dialinecolor}
\pgfsetstrokeopacity{1.000000}
\draw (-4.781161\du,8.385587\du)--(-4.781161\du,5.658188\du);
}
\pgfsetlinewidth{0.100000\du}
\pgfsetdash{}{0pt}
\pgfsetbuttcap
{
\definecolor{diafillcolor}{rgb}{0.000000, 0.000000, 0.000000}
\pgfsetfillcolor{diafillcolor}
\pgfsetfillopacity{1.000000}
% was here!!!
\pgfsetarrowsstart{stealth}
\definecolor{dialinecolor}{rgb}{0.000000, 0.000000, 0.000000}
\pgfsetstrokecolor{dialinecolor}
\pgfsetstrokeopacity{1.000000}
\draw (-3.905082\du,8.421403\du)--(-3.905082\du,5.694004\du);
}
\pgfsetlinewidth{0.100000\du}
\pgfsetdash{}{0pt}
\pgfsetbuttcap
{
\definecolor{diafillcolor}{rgb}{0.000000, 0.000000, 0.000000}
\pgfsetfillcolor{diafillcolor}
\pgfsetfillopacity{1.000000}
% was here!!!
\pgfsetarrowsstart{stealth}
\definecolor{dialinecolor}{rgb}{0.000000, 0.000000, 0.000000}
\pgfsetstrokecolor{dialinecolor}
\pgfsetstrokeopacity{1.000000}
\draw (-16.824827\du,9.828245\du)--(-7.243893\du,9.828245\du);
}
\pgfsetlinewidth{0.100000\du}
\pgfsetdash{}{0pt}
\pgfsetbuttcap
{
\definecolor{diafillcolor}{rgb}{0.000000, 0.000000, 0.000000}
\pgfsetfillcolor{diafillcolor}
\pgfsetfillopacity{1.000000}
% was here!!!
\pgfsetarrowsend{stealth}
\definecolor{dialinecolor}{rgb}{0.000000, 0.000000, 0.000000}
\pgfsetstrokecolor{dialinecolor}
\pgfsetstrokeopacity{1.000000}
\draw (-16.815066\du,8.994142\du)--(-7.234132\du,8.994142\du);
}
\end{tikzpicture}
}
	\caption{
		By embedding the homeserver in the client application, two clients can communicate with each other without any external homeserver.
	}%
	\label{fig:p2p-embed}
\end{figure}

\section{Problem Statement}\label{sec:problem_statement}

From the analysis we have learned that \ac{P2P} Matrix is possible by embedding a homeserver into a client.
We also learned that we must carefully monitor the use of battery and storage space when developing a mobile application.

We expand the initial problem statement from Section~\ref{subsec:initial_problem_statement} with the following questions:\\
\begin{itemize}
	\item What are the minimum requirements for running peer-to-peer Matrix on Android?
	\item How well does such a solution respect mobile constraints?
	\item What are the properties of a competitive solution, with regard to walled garden alternatives?
\end{itemize}