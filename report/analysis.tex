\chapter{Analysis}
We begin the analysis by examining the Matrix Open Standard.

\section{The Matrix Standard}
The Matrix Standard defines several \ac{API}s.
Each of these \ac{API}s is expected to communicate with \ac{JSON} over \ac{REST}.
To be more precise, the following \ac{API}s are defined:
\begin{itemize}
    \item \textbf{Client-Server}: A \textit{homeserver} interacting with a \textit{client}. In peer-to-peer Matrix, the \textit{client} and \textit{homeserver} are both on-device.
    \item \textbf{Server-Server}: Two or more \textit{homeservers} interacting. The servers synchronize state via \textit{federation}.
    \item \textbf{Application Service}: Interoperability and extensibility is largely achieved through application services, which can \textit{bridge} external chat protocols to Matrix, allowing e.g. interoperability between \ac{IRC} and Matrix.
    \item \textbf{Identity Service}: Identity services enable external identities, such as email, mobile numbers, or other services to be attached to a Matrix user.
    \item \textbf{Push Gateway}: The way push notifications are implemented for Matrix events.
\end{itemize}

Each homeserver needs to be registered with a DNS entry, by which they are currently identified.
A user is registered under a homeserver, and identified thereby, e.g. \texttt{@user:homeserver.domain}.
The specification is meant to change, i.e. it was not designed as final, and it has an established process for changes.
This process is defined in \cite{matrix_org_spec_changes}.
In order to account for a changing standard, a versioning system was established by the Matrix foundation.
This standard is found in \cite{matrix_org_spec}, and establishes the foundation for interoperability between clients and chat rooms of varying versions.

\subsection{Dendrite}
While Synapse is the reference implementation, progress on the peer-to-peer implementation is being made on Dendrite, the second-generation homeserver.
Dendrite is written in Go, which typically compiles to native binaries.

\subsection{RiotX}

\section{Problem Statement}
\begin{itemize}
    \item TODO
\end{itemize}
