\chapter{Conclusion}
Turning back to the problem statement introduced in Section~\ref{sec:problem_statement}, we now consider how well the project answers the questions therein.

\paragraph\ The first question:
\begin{quote}
	\textit{What are the minimum requirements for running peer-to-peer Matrix on Android?}
\end{quote}
From our implementation, we determine that the minimum requirements for running peer-to-peer Matrix on Android are \textit{implementing the Matrix client and server \ac{API}s in a single application, where the server \ac{API}s include peer-to-peer networking}.
Given the current state of the ecosystem, this task is becoming less and less difficult.
As a result of our contribution, an updated mobile-compatible version of Dendrite is now hosted in the main version of Dendrite at \github{matrix-org/dendrite}.

\paragraph\ The second question:
\begin{quote}
	\textit{How well does such a solution respect mobile constraints?}
\end{quote}

We evaluated the solution with regard to two mobile constraints in Chapter~\ref{chp:a_critical_evaluation}.
\todo{Battery life!}.
The solution works well with limited storage space given two limitations: The user does not participate in large chat rooms, and limits the use of space intensive media, such as voice snippets, images, video or large files.
We say \textit{large} chat rooms being well aware that it is a loose term, as the user may act with a relatively large degree of freedom in this matter.
The larger the rooms the user participates in, the higher the chance that the user runs out of space.
In Section~\ref{sec:fog_computing}, we suggest an alternative method: fog computing.
This solution allows similar functionality without sacrificing space, and may improve the energy consumption vs\@. an app that uses a centralized server.

\paragraph\ The third question:
\begin{quote}
	\textit{What are the properties of a competitive solution, with regard to walled garden alternatives?}
\end{quote}

A competitive solution would likely need the following features, which we were not able to implement:
\begin{itemize}
	\item{
	      \textbf{Federation with Centralized Servers}:
	      Federating with non-P2P Matrix servers.
	      This allows strategies such as having a resident homeserver, as mentioned in the discussion.\todo{Mention resident strategy in discussion.}
	      The main issue with this strategy is that the current Matrix standard requires a DNS entry, which is not tractable for peer-to-peer servers.
	      }
	\item{
	      \textbf{Internet Peer-to-Peer}:
	      We currently only support peer discovery on the devices' own network.
	      A complete solution would support discovery in a larger area.
	      This is a basic requirement which is easily fulfilled by chat applications with a centralized architecture.
	      }
	\item{
	      \textbf{Thin Client-Server API}:
	      We combined a full Dendrite server with the Riot client.
	      The Client-Server API is designed to be flexible, but we don't need that flexibility.
	      Instead of wasting resources on HTTP requests, we could allow the client and server to communicate more directly.
	      For example, the client and server could be written in the same language, and working with the same data structures in memory.
	      }
\end{itemize}