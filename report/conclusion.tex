\chapter{Conclusion}

From our implementation, we determine that the minimum requirements for running peer-to-peer Matrix on Android are \textit{implementing the Matrix client and server \ac{API}s in a single application, where the server \ac{API}s include peer-to-peer networking}.
Given the current state of the ecosystem, this task is becoming less and less difficult.
We helped integrate the peer-to-peer version of Dendrite into the master branch, where it will receive updates along with the rest of Dendrite.

However, the solution we created was not complete.
Notably, the following features were missing:
\begin{itemize}
	\item{
	      \textbf{Federation with Centralized Servers}:
	      \github{lihram/server} does not support federating with non-P2P Matrix servers.
	      This prevents strategies such as having a resident homeserver, as mentioned in the discussion.\todo{Mention resident strategy in discussion.}
	      The main issue with this strategy is that the current Matrix standard requires a DNS entry, which is not tractable for peer-to-peer servers.
	      }
	\item{
	      \textbf{Internet Peer-to-Peer}:
	      We currently only support peer discovery on the devices' own network.
	      A complete solution would support discovery in a larger area.
	      This is a basic requirement which is easily fulfilled by chat applications with a centralized architecture.
	      }
	\item{
	      \textbf{Thin Client-Server API}:
	      We combined a full Dendrite server with the Riot client.
	      The Client-Server API is designed to be flexible, but we don't need that flexibility.
	      Instead of wasting resources on HTTP requests, we could allow the client and server to communicate more directly.
	      For example, the client and server could be written in the same language, and working with the same data structures in memory.
	      }
\end{itemize}