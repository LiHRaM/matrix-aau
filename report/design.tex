\chapter{Design Constraints}\label{chp:design}
This chapter consists of an initial section describing the constraints of the design, followed by a dilemma, and concluding with a design overview.
The dilemma motivates a choice of technologies which impacts the design.
Finally, we conclude the chapter by providing the final design overview.

\section{Constraints}
The design of the system happens within the context of several larger systems.
\begin{itemize}
	\item{
	      \textbf{The Matrix Standard}:
	      The most fundamental constraint is the Matrix standard.
	      Both the client and the server are designed with the Matrix protocols in mind.
	      The goal of this project is not to modify these protocols, but rather to see whether an extension of it is viable.
	      }
	\item{
	      \textbf{Homeserver Architecture}:
	      The homeservers are designed to implement the Matrix protocol, but do so differently.
	      The Synapse approach is to provide a large self-contained unit, whereas Dendrite provides these services in a more composable way.
	      Either way, these are large projects which at the very least require a good understanding of both the Matrix protocol and the design of the components which we modify or extend.
	      }
	\item{
	      \textbf{Android Architecture}:
	      Both Riot and RiotX are built for the Android operating system.
	      The Google developer documentation for Android has guidelines for how an Android application should be developed.
	      This includes specific design principles and the Android \ac{SDK}, which provides a way for the application to interact with the operating system.
	      }
\end{itemize}

Figure~\ref{fig:design_constraints} illustrates the layering of these constraints.
The focus of our design is to ensure that these components interact correctly under new circumstances, i.e.~as a single application containing both the client and the server.
For example, we do not intend to modify the Client-Server communication.

\begin{figure}
	\centering
	\resizebox{!}{!}{% Graphic for TeX using PGF
% Title: /home/lihram/git/github.com/lihram/matrix-aau/report/graphics/design-constraints.dia
% Creator: Dia v0.97+git
% CreationDate: Sat May 16 12:55:48 2020
% For: lihram
% \usepackage{tikz}
% The following commands are not supported in PSTricks at present
% We define them conditionally, so when they are implemented,
% this pgf file will use them.
\ifx\du\undefined
  \newlength{\du}
\fi
\setlength{\du}{15\unitlength}
\begin{tikzpicture}[even odd rule]
\pgftransformxscale{1.000000}
\pgftransformyscale{-1.000000}
\definecolor{dialinecolor}{rgb}{0.000000, 0.000000, 0.000000}
\pgfsetstrokecolor{dialinecolor}
\pgfsetstrokeopacity{1.000000}
\definecolor{diafillcolor}{rgb}{1.000000, 1.000000, 1.000000}
\pgfsetfillcolor{diafillcolor}
\pgfsetfillopacity{1.000000}
\pgfsetlinewidth{0.100000\du}
\pgfsetdash{}{0pt}
\pgfsetmiterjoin
{\pgfsetcornersarced{\pgfpoint{0.000000\du}{0.000000\du}}\definecolor{diafillcolor}{rgb}{1.000000, 1.000000, 1.000000}
\pgfsetfillcolor{diafillcolor}
\pgfsetfillopacity{1.000000}
\fill (10.198800\du,16.060000\du)--(10.198800\du,20.510000\du)--(33.648800\du,20.510000\du)--(33.648800\du,16.060000\du)--cycle;
}{\pgfsetcornersarced{\pgfpoint{0.000000\du}{0.000000\du}}\definecolor{dialinecolor}{rgb}{0.000000, 0.000000, 0.000000}
\pgfsetstrokecolor{dialinecolor}
\pgfsetstrokeopacity{1.000000}
\draw (10.198800\du,16.060000\du)--(10.198800\du,20.510000\du)--(33.648800\du,20.510000\du)--(33.648800\du,16.060000\du)--cycle;
}% setfont left to latex
\definecolor{dialinecolor}{rgb}{0.000000, 0.000000, 0.000000}
\pgfsetstrokecolor{dialinecolor}
\pgfsetstrokeopacity{1.000000}
\definecolor{diafillcolor}{rgb}{0.000000, 0.000000, 0.000000}
\pgfsetfillcolor{diafillcolor}
\pgfsetfillopacity{1.000000}
\node[anchor=base,inner sep=0pt, outer sep=0pt,color=dialinecolor] at (21.923800\du,18.529966\du){The Matrix Standard};
\pgfsetlinewidth{0.100000\du}
\pgfsetdash{}{0pt}
\pgfsetmiterjoin
{\pgfsetcornersarced{\pgfpoint{0.000000\du}{0.000000\du}}\definecolor{diafillcolor}{rgb}{1.000000, 1.000000, 1.000000}
\pgfsetfillcolor{diafillcolor}
\pgfsetfillopacity{1.000000}
\fill (13.346300\du,11.610000\du)--(13.346300\du,16.060000\du)--(30.501300\du,16.060000\du)--(30.501300\du,11.610000\du)--cycle;
}{\pgfsetcornersarced{\pgfpoint{0.000000\du}{0.000000\du}}\definecolor{dialinecolor}{rgb}{0.000000, 0.000000, 0.000000}
\pgfsetstrokecolor{dialinecolor}
\pgfsetstrokeopacity{1.000000}
\draw (13.346300\du,11.610000\du)--(13.346300\du,16.060000\du)--(30.501300\du,16.060000\du)--(30.501300\du,11.610000\du)--cycle;
}% setfont left to latex
\definecolor{dialinecolor}{rgb}{0.000000, 0.000000, 0.000000}
\pgfsetstrokecolor{dialinecolor}
\pgfsetstrokeopacity{1.000000}
\definecolor{diafillcolor}{rgb}{0.000000, 0.000000, 0.000000}
\pgfsetfillcolor{diafillcolor}
\pgfsetfillopacity{1.000000}
\node[anchor=base,inner sep=0pt, outer sep=0pt,color=dialinecolor] at (21.923800\du,14.079966\du){Homeserver Architecture};
\pgfsetlinewidth{0.100000\du}
\pgfsetdash{}{0pt}
\pgfsetmiterjoin
{\pgfsetcornersarced{\pgfpoint{0.000000\du}{0.000000\du}}\definecolor{diafillcolor}{rgb}{1.000000, 1.000000, 1.000000}
\pgfsetfillcolor{diafillcolor}
\pgfsetfillopacity{1.000000}
\fill (16.646300\du,7.160000\du)--(16.646300\du,11.610000\du)--(27.201300\du,11.610000\du)--(27.201300\du,7.160000\du)--cycle;
}{\pgfsetcornersarced{\pgfpoint{0.000000\du}{0.000000\du}}\definecolor{dialinecolor}{rgb}{0.000000, 0.000000, 0.000000}
\pgfsetstrokecolor{dialinecolor}
\pgfsetstrokeopacity{1.000000}
\draw (16.646300\du,7.160000\du)--(16.646300\du,11.610000\du)--(27.201300\du,11.610000\du)--(27.201300\du,7.160000\du)--cycle;
}% setfont left to latex
\definecolor{dialinecolor}{rgb}{0.000000, 0.000000, 0.000000}
\pgfsetstrokecolor{dialinecolor}
\pgfsetstrokeopacity{1.000000}
\definecolor{diafillcolor}{rgb}{0.000000, 0.000000, 0.000000}
\pgfsetfillcolor{diafillcolor}
\pgfsetfillopacity{1.000000}
\node[anchor=base,inner sep=0pt, outer sep=0pt,color=dialinecolor] at (21.923800\du,9.629966\du){Android Architecture};
\end{tikzpicture}
}
	\caption{%
		The three layers of constraints with which we work.
		The Matrix Standard is fundamental to the design, followed by the homeserver architecture and lastly the android ecosystem.
	}%
	\label{fig:design_constraints}
\end{figure}

\chapter{Design Dilemma and Resolution}\label{chp:design_dilemma}

\section{Dilemma: Stable versus Bleeding Edge}
The first dilemma consists of choosing a homeserver and a client to work with.
We must consider the advantages and disadvantages of both the older, more stable Synapse and Riot versus the newer Dendrite and RiotX.
There are many other clients and homeservers which we could consider as well, many of which are listed on the Matrix website.
However, as described in the analysis, these alternatives are not officially supported and are therefore risky choices.

\subsection{Approach}
We begin by considering the two homeservers described in Section~\ref{sec:official_homeservers}.
Afterwards we repeat the procedure for the two clients described in Section~\ref{sec:official_clients}.
We estimate the work required for each homeserver to support the minimum requirements for the project, which are:
\begin{itemize}
	\item{
	      \textbf{Peer-to-Peer}:
	      The server must be able to participate in a peer-to-peer network, by peer discovery and signaling its presence.
	      }
	\item{
	      \textbf{Android Packaging}:
	      Riot and RiotX use the Gradle\footnote{https://gradle.org} build system, which transforms source program into an Android application.
	      We therefore need a method to integrate the homeserver in this build system in order to package it with the application.
	      This also allows the clients to start and stop the servers as necessary.
	      }
\end{itemize}

Table~\ref{tab:homeserver_comparison} describes the result of this comparison, which we now describe in further detail.

\begin{table}
	\center{}
	\begin{tabular}{lccc}
		\textbf{Server} & \multicolumn{1}{l}{\textbf{Peer-to-Peer}} & \multicolumn{1}{l}{\textbf{Android Packaging}} & \multicolumn{1}{l}{\textbf{Embedded Database Support}} \\ \toprule
		Synapse         & No                                        & No                                             & Yes                                                    \\
		Dendrite        & Yes                                       & Yes                                            & Yes\footnotemark\
	\end{tabular}
	\caption{An overview of the fitness of the two homeservers for use as embedded servers.}%
	\label{tab:homeserver_comparison}
\end{table}
\footnotetext{
	The version of Dendrite with peer-to-peer networking uses an older version of Dendrite which does not support embedded databases.
	The latest Dendrite does support embedded databases, however.
}

\subsubsection{Synapse}
Synapse is a natural choice for many projects which need a Matrix homeserver.
It is the official, recommended homeserver for personal and commercial use, and is thoroughly tested.
It is the reference implementation, and is stable and feature complete.
On the other hand, it does not support either peer-to-peer networking or Android packaging.
If we were to choose Synapse, we would have to implement all the peer-to-peer logic which is already implemented in Dendrite.
Additionally, as there are no Python implementations of libp2p, Synapse would have to use a different set of libraries.
One could also choose to invest in libp2p for Python and implement those libraries as well, but that would require an even greater amount of work.
In terms of Android packaging, there are no official solutions for Python, but there is an unofficial solution known as BeeWare\footnotemark.
\footnotetext{https://beeware.org}
Unfortunately, as its documentation states\footnotemark, the build tools are incompatible with the latest changes in the Android ecosystem.
\footnotetext{https://docs.beeware.org/en/latest/tutorial/tutorial-5/index.html}
It would be a major effort to use Synapse as the embedded homeserver at this time.

\subsubsection{Dendrite}
Dendrite is not recommended for serious use yet.
It lacks the stability and features that Synapse has.
For example, it only supports Client-Server API up to version v0.3.0, whereas the latest stable is v0.6.0.
It is actively being changed, and is overall a very new project compared to the older and more stable Synapse.
However, through the \texttt{Gomobile} project, it can be packaged for Android, and there is already a version which contains a peer-to-peer implementation.
Peer-to-peer Dendrite does not support Android completely though, as it uses an older version of Dendrite, which does not support embedded databases yet.
This is necessary to store the message metadata in the app that Dendrite needs.
In summary, Dendrite has two of the required features, but requires updating in order to obtain the third.

\subsubsection{Riot}
Riot for Android is conceptually similar to Synapse, in the sense that it is an older, more stable and mature application than its newer alternative.
It supports both Synapse and Dendrite.
However, it is not under active development anymore, as described in Section~\ref{sec:official_clients}.
Instead, they point new contributors to RiotX instead, which is now the main focus for Android development.

\subsubsection{RiotX}
RiotX is similar to Dendrite, in that it is the official next generation client.
Currently, it is in beta stage, and does not support all the features that Riot does.
Notably, it does not support older versions of the Client-Server API, requiring versions v0.5.0 and v0.6.0.
This makes it incompatible with Dendrite at this time.
Choosing RiotX would therefore require work to bring Dendrite up to those versions.

\subsection{Resolution}
We choose Dendrite as the embedded homeserver, based on the criteria listed in Table~\ref{tab:homeserver_comparison}, and Riot for Android as the client, as RiotX does not at this time support Dendrite as an embedded homeserver.

\section{Design Overview}
We design the combination of Dendrite and Riot into a single Android App, as illustrated in Figure~\ref{fig:p2p-embed}.
The flow of the design is as follows:

As Riot launches, it initializes the Dendrite server in a separate thread.
This allows Riot and Dendrite to run at the same time.
When Dendrite runs for the first time, it creates databases for message metadata, user data, and security keys.
Once it is done, it notifies Riot through a synchronization mechanism, and begins listening for incoming requests from the client and other servers.
Meanwhile, Riot waits for the address before it can continue.
Once it receives the address, it saves it and launches normally.
This design is illustrated in Figure~\ref{fig:design_overview}.

\begin{figure}
	\centering
	\resizebox{0.5\linewidth}{!}{% Graphic for TeX using PGF
% Title: /home/lihram/git/github.com/lihram/matrix-aau/report/graphics/design-overview.dia
% Creator: Dia v0.97+git
% CreationDate: Sat May 16 15:14:03 2020
% For: lihram
% \usepackage{tikz}
% The following commands are not supported in PSTricks at present
% We define them conditionally, so when they are implemented,
% this pgf file will use them.
\ifx\du\undefined
  \newlength{\du}
\fi
\setlength{\du}{15\unitlength}
\begin{tikzpicture}[even odd rule]
\pgftransformxscale{1.000000}
\pgftransformyscale{-1.000000}
\definecolor{dialinecolor}{rgb}{0.000000, 0.000000, 0.000000}
\pgfsetstrokecolor{dialinecolor}
\pgfsetstrokeopacity{1.000000}
\definecolor{diafillcolor}{rgb}{1.000000, 1.000000, 1.000000}
\pgfsetfillcolor{diafillcolor}
\pgfsetfillopacity{1.000000}
\pgfsetlinewidth{0.100000\du}
\pgfsetdash{}{0pt}
\pgfsetmiterjoin
\definecolor{diafillcolor}{rgb}{1.000000, 1.000000, 1.000000}
\pgfsetfillcolor{diafillcolor}
\pgfsetfillopacity{1.000000}
\pgfpathellipse{\pgfpoint{22.003364\du}{5.776682\du}}{\pgfpoint{3.203364\du}{0\du}}{\pgfpoint{0\du}{2.176682\du}}
\pgfusepath{fill}
\definecolor{dialinecolor}{rgb}{0.000000, 0.000000, 0.000000}
\pgfsetstrokecolor{dialinecolor}
\pgfsetstrokeopacity{1.000000}
\pgfpathellipse{\pgfpoint{22.003364\du}{5.776682\du}}{\pgfpoint{3.203364\du}{0\du}}{\pgfpoint{0\du}{2.176682\du}}
\pgfusepath{stroke}
% setfont left to latex
\definecolor{dialinecolor}{rgb}{0.000000, 0.000000, 0.000000}
\pgfsetstrokecolor{dialinecolor}
\pgfsetstrokeopacity{1.000000}
\definecolor{diafillcolor}{rgb}{0.000000, 0.000000, 0.000000}
\pgfsetfillcolor{diafillcolor}
\pgfsetfillopacity{1.000000}
\node[anchor=base,inner sep=0pt, outer sep=0pt,color=dialinecolor] at (22.003364\du,6.021648\du){Program Start};
\pgfsetlinewidth{0.100000\du}
\pgfsetdash{}{0pt}
\pgfsetmiterjoin
{\pgfsetcornersarced{\pgfpoint{0.000000\du}{0.000000\du}}\definecolor{diafillcolor}{rgb}{1.000000, 1.000000, 1.000000}
\pgfsetfillcolor{diafillcolor}
\pgfsetfillopacity{1.000000}
\fill (26.000000\du,9.200000\du)--(26.000000\du,12.450000\du)--(32.000000\du,12.450000\du)--(32.000000\du,9.200000\du)--cycle;
}{\pgfsetcornersarced{\pgfpoint{0.000000\du}{0.000000\du}}\definecolor{dialinecolor}{rgb}{0.000000, 0.000000, 0.000000}
\pgfsetstrokecolor{dialinecolor}
\pgfsetstrokeopacity{1.000000}
\draw (26.000000\du,9.200000\du)--(26.000000\du,12.450000\du)--(32.000000\du,12.450000\du)--(32.000000\du,9.200000\du)--cycle;
}% setfont left to latex
\definecolor{dialinecolor}{rgb}{0.000000, 0.000000, 0.000000}
\pgfsetstrokecolor{dialinecolor}
\pgfsetstrokeopacity{1.000000}
\definecolor{diafillcolor}{rgb}{0.000000, 0.000000, 0.000000}
\pgfsetfillcolor{diafillcolor}
\pgfsetfillopacity{1.000000}
\node[anchor=base,inner sep=0pt, outer sep=0pt,color=dialinecolor] at (29.000000\du,11.069966\du){Dendrite};
\pgfsetlinewidth{0.100000\du}
\pgfsetdash{}{0pt}
\pgfsetbuttcap
{
\definecolor{diafillcolor}{rgb}{0.000000, 0.000000, 0.000000}
\pgfsetfillcolor{diafillcolor}
\pgfsetfillopacity{1.000000}
% was here!!!
\definecolor{dialinecolor}{rgb}{0.000000, 0.000000, 0.000000}
\pgfsetstrokecolor{dialinecolor}
\pgfsetstrokeopacity{1.000000}
\draw (24.239776\du,7.390332\du)--(26.678179\du,9.149725\du);
}
\definecolor{dialinecolor}{rgb}{0.000000, 0.000000, 0.000000}
\pgfsetstrokecolor{dialinecolor}
\pgfsetstrokeopacity{1.000000}
\draw (24.239776\du,7.390332\du)--(26.182041\du,8.791744\du);
\pgfsetlinewidth{0.100000\du}
\pgfsetdash{}{0pt}
\pgfsetmiterjoin
\definecolor{diafillcolor}{rgb}{1.000000, 1.000000, 1.000000}
\pgfsetfillcolor{diafillcolor}
\pgfsetfillopacity{1.000000}
\fill (26.035760\du,8.994480\du)--(26.587513\du,9.084306\du)--(26.328322\du,8.589008\du)--cycle;
\definecolor{dialinecolor}{rgb}{0.000000, 0.000000, 0.000000}
\pgfsetstrokecolor{dialinecolor}
\pgfsetstrokeopacity{1.000000}
\draw (26.035760\du,8.994480\du)--(26.587513\du,9.084306\du)--(26.328322\du,8.589008\du)--cycle;
\pgfsetlinewidth{0.100000\du}
\pgfsetdash{}{0pt}
\pgfsetbuttcap
\pgfsetmiterjoin
\pgfsetlinewidth{0.100000\du}
\pgfsetbuttcap
\pgfsetmiterjoin
\pgfsetdash{}{0pt}
\definecolor{diafillcolor}{rgb}{1.000000, 1.000000, 1.000000}
\pgfsetfillcolor{diafillcolor}
\pgfsetfillopacity{1.000000}
\definecolor{dialinecolor}{rgb}{0.000000, 0.000000, 0.000000}
\pgfsetstrokecolor{dialinecolor}
\pgfsetstrokeopacity{1.000000}
\pgfpathmoveto{\pgfpoint{19.400000\du}{9.200000\du}}
\pgfpathlineto{\pgfpoint{22.866667\du}{9.200000\du}}
\pgfpathcurveto{\pgfpoint{23.823961\du}{9.200000\du}}{\pgfpoint{24.600000\du}{9.916344\du}}{\pgfpoint{24.600000\du}{10.800000\du}}
\pgfpathcurveto{\pgfpoint{24.600000\du}{11.683656\du}}{\pgfpoint{23.823961\du}{12.400000\du}}{\pgfpoint{22.866667\du}{12.400000\du}}
\pgfpathlineto{\pgfpoint{19.400000\du}{12.400000\du}}
\pgfpathlineto{\pgfpoint{19.400000\du}{9.200000\du}}
\pgfpathclose
\pgfusepath{fill,stroke}
% setfont left to latex
\definecolor{dialinecolor}{rgb}{0.000000, 0.000000, 0.000000}
\pgfsetstrokecolor{dialinecolor}
\pgfsetstrokeopacity{1.000000}
\definecolor{diafillcolor}{rgb}{0.000000, 0.000000, 0.000000}
\pgfsetfillcolor{diafillcolor}
\pgfsetfillopacity{1.000000}
\node[anchor=base east,inner sep=0pt, outer sep=0pt,color=dialinecolor] at (22.766667\du,11.044967\du){Wait};
% setfont left to latex
\definecolor{dialinecolor}{rgb}{0.000000, 0.000000, 0.000000}
\pgfsetstrokecolor{dialinecolor}
\pgfsetstrokeopacity{1.000000}
\definecolor{diafillcolor}{rgb}{0.000000, 0.000000, 0.000000}
\pgfsetfillcolor{diafillcolor}
\pgfsetfillopacity{1.000000}
\node[anchor=base west,inner sep=0pt,outer sep=0pt,color=dialinecolor] at (22.000000\du,10.800000\du){};
\pgfsetlinewidth{0.100000\du}
\pgfsetdash{}{0pt}
\pgfsetbuttcap
{
\definecolor{diafillcolor}{rgb}{0.000000, 0.000000, 0.000000}
\pgfsetfillcolor{diafillcolor}
\pgfsetfillopacity{1.000000}
% was here!!!
\definecolor{dialinecolor}{rgb}{0.000000, 0.000000, 0.000000}
\pgfsetstrokecolor{dialinecolor}
\pgfsetstrokeopacity{1.000000}
\draw (22.001876\du,7.998912\du)--(22.001105\du,9.150497\du);
}
\definecolor{dialinecolor}{rgb}{0.000000, 0.000000, 0.000000}
\pgfsetstrokecolor{dialinecolor}
\pgfsetstrokeopacity{1.000000}
\draw (22.001876\du,7.998912\du)--(22.001514\du,8.538694\du);
\pgfsetlinewidth{0.100000\du}
\pgfsetdash{}{0pt}
\pgfsetmiterjoin
\definecolor{diafillcolor}{rgb}{1.000000, 1.000000, 1.000000}
\pgfsetfillcolor{diafillcolor}
\pgfsetfillopacity{1.000000}
\fill (21.751514\du,8.538527\du)--(22.001179\du,9.038694\du)--(22.251514\du,8.538862\du)--cycle;
\definecolor{dialinecolor}{rgb}{0.000000, 0.000000, 0.000000}
\pgfsetstrokecolor{dialinecolor}
\pgfsetstrokeopacity{1.000000}
\draw (21.751514\du,8.538527\du)--(22.001179\du,9.038694\du)--(22.251514\du,8.538862\du)--cycle;
\pgfsetlinewidth{0.100000\du}
\pgfsetdash{}{0pt}
\pgfsetmiterjoin
\definecolor{diafillcolor}{rgb}{1.000000, 1.000000, 1.000000}
\pgfsetfillcolor{diafillcolor}
\pgfsetfillopacity{1.000000}
\pgfpathellipse{\pgfpoint{22.003364\du}{16.176682\du}}{\pgfpoint{3.203364\du}{0\du}}{\pgfpoint{0\du}{2.176682\du}}
\pgfusepath{fill}
\definecolor{dialinecolor}{rgb}{0.000000, 0.000000, 0.000000}
\pgfsetstrokecolor{dialinecolor}
\pgfsetstrokeopacity{1.000000}
\pgfpathellipse{\pgfpoint{22.003364\du}{16.176682\du}}{\pgfpoint{3.203364\du}{0\du}}{\pgfpoint{0\du}{2.176682\du}}
\pgfusepath{stroke}
% setfont left to latex
\definecolor{dialinecolor}{rgb}{0.000000, 0.000000, 0.000000}
\pgfsetstrokecolor{dialinecolor}
\pgfsetstrokeopacity{1.000000}
\definecolor{diafillcolor}{rgb}{0.000000, 0.000000, 0.000000}
\pgfsetfillcolor{diafillcolor}
\pgfsetfillopacity{1.000000}
\node[anchor=base,inner sep=0pt, outer sep=0pt,color=dialinecolor] at (22.003364\du,16.421648\du){Synchronize};
\pgfsetlinewidth{0.100000\du}
\pgfsetdash{}{0pt}
\pgfsetbuttcap
{
\definecolor{diafillcolor}{rgb}{0.000000, 0.000000, 0.000000}
\pgfsetfillcolor{diafillcolor}
\pgfsetfillopacity{1.000000}
% was here!!!
\definecolor{dialinecolor}{rgb}{0.000000, 0.000000, 0.000000}
\pgfsetstrokecolor{dialinecolor}
\pgfsetstrokeopacity{1.000000}
\draw (22.001032\du,12.449365\du)--(22.001971\du,13.949743\du);
}
\definecolor{dialinecolor}{rgb}{0.000000, 0.000000, 0.000000}
\pgfsetstrokecolor{dialinecolor}
\pgfsetstrokeopacity{1.000000}
\draw (22.001032\du,12.449365\du)--(22.001588\du,13.337940\du);
\pgfsetlinewidth{0.100000\du}
\pgfsetdash{}{0pt}
\pgfsetmiterjoin
\definecolor{diafillcolor}{rgb}{1.000000, 1.000000, 1.000000}
\pgfsetfillcolor{diafillcolor}
\pgfsetfillopacity{1.000000}
\fill (21.751588\du,13.338096\du)--(22.001901\du,13.837940\du)--(22.251588\du,13.337784\du)--cycle;
\definecolor{dialinecolor}{rgb}{0.000000, 0.000000, 0.000000}
\pgfsetstrokecolor{dialinecolor}
\pgfsetstrokeopacity{1.000000}
\draw (21.751588\du,13.338096\du)--(22.001901\du,13.837940\du)--(22.251588\du,13.337784\du)--cycle;
\pgfsetlinewidth{0.100000\du}
\pgfsetdash{}{0pt}
\pgfsetbuttcap
{
\definecolor{diafillcolor}{rgb}{0.000000, 0.000000, 0.000000}
\pgfsetfillcolor{diafillcolor}
\pgfsetfillopacity{1.000000}
% was here!!!
\definecolor{dialinecolor}{rgb}{0.000000, 0.000000, 0.000000}
\pgfsetstrokecolor{dialinecolor}
\pgfsetstrokeopacity{1.000000}
\draw (26.811843\du,12.498707\du)--(24.170169\du,14.519307\du);
}
\definecolor{dialinecolor}{rgb}{0.000000, 0.000000, 0.000000}
\pgfsetstrokecolor{dialinecolor}
\pgfsetstrokeopacity{1.000000}
\draw (26.811843\du,12.498707\du)--(24.656116\du,14.147609\du);
\pgfsetlinewidth{0.100000\du}
\pgfsetdash{}{0pt}
\pgfsetmiterjoin
\definecolor{diafillcolor}{rgb}{1.000000, 1.000000, 1.000000}
\pgfsetfillcolor{diafillcolor}
\pgfsetfillopacity{1.000000}
\fill (24.504230\du,13.949038\du)--(24.258973\du,14.451381\du)--(24.808002\du,14.346181\du)--cycle;
\definecolor{dialinecolor}{rgb}{0.000000, 0.000000, 0.000000}
\pgfsetstrokecolor{dialinecolor}
\pgfsetstrokeopacity{1.000000}
\draw (24.504230\du,13.949038\du)--(24.258973\du,14.451381\du)--(24.808002\du,14.346181\du)--cycle;
\pgfsetlinewidth{0.100000\du}
\pgfsetdash{}{0pt}
\pgfsetmiterjoin
\definecolor{diafillcolor}{rgb}{1.000000, 1.000000, 1.000000}
\pgfsetfillcolor{diafillcolor}
\pgfsetfillopacity{1.000000}
\pgfpathellipse{\pgfpoint{22.003364\du}{21.976682\du}}{\pgfpoint{3.203364\du}{0\du}}{\pgfpoint{0\du}{2.176682\du}}
\pgfusepath{fill}
\definecolor{dialinecolor}{rgb}{0.000000, 0.000000, 0.000000}
\pgfsetstrokecolor{dialinecolor}
\pgfsetstrokeopacity{1.000000}
\pgfpathellipse{\pgfpoint{22.003364\du}{21.976682\du}}{\pgfpoint{3.203364\du}{0\du}}{\pgfpoint{0\du}{2.176682\du}}
\pgfusepath{stroke}
% setfont left to latex
\definecolor{dialinecolor}{rgb}{0.000000, 0.000000, 0.000000}
\pgfsetstrokecolor{dialinecolor}
\pgfsetstrokeopacity{1.000000}
\definecolor{diafillcolor}{rgb}{0.000000, 0.000000, 0.000000}
\pgfsetfillcolor{diafillcolor}
\pgfsetfillopacity{1.000000}
\node[anchor=base,inner sep=0pt, outer sep=0pt,color=dialinecolor] at (22.003364\du,22.221648\du){Store Address};
\pgfsetlinewidth{0.100000\du}
\pgfsetdash{}{0pt}
\pgfsetbuttcap
{
\definecolor{diafillcolor}{rgb}{0.000000, 0.000000, 0.000000}
\pgfsetfillcolor{diafillcolor}
\pgfsetfillopacity{1.000000}
% was here!!!
\definecolor{dialinecolor}{rgb}{0.000000, 0.000000, 0.000000}
\pgfsetstrokecolor{dialinecolor}
\pgfsetstrokeopacity{1.000000}
\draw (22.003364\du,18.403367\du)--(22.003364\du,19.749997\du);
}
\definecolor{dialinecolor}{rgb}{0.000000, 0.000000, 0.000000}
\pgfsetstrokecolor{dialinecolor}
\pgfsetstrokeopacity{1.000000}
\draw (22.003364\du,18.403367\du)--(22.003364\du,19.138194\du);
\pgfsetlinewidth{0.100000\du}
\pgfsetdash{}{0pt}
\pgfsetmiterjoin
\definecolor{diafillcolor}{rgb}{1.000000, 1.000000, 1.000000}
\pgfsetfillcolor{diafillcolor}
\pgfsetfillopacity{1.000000}
\fill (21.753364\du,19.138194\du)--(22.003364\du,19.638194\du)--(22.253364\du,19.138194\du)--cycle;
\definecolor{dialinecolor}{rgb}{0.000000, 0.000000, 0.000000}
\pgfsetstrokecolor{dialinecolor}
\pgfsetstrokeopacity{1.000000}
\draw (21.753364\du,19.138194\du)--(22.003364\du,19.638194\du)--(22.253364\du,19.138194\du)--cycle;
\pgfsetlinewidth{0.100000\du}
\pgfsetdash{}{0pt}
\pgfsetmiterjoin
\definecolor{diafillcolor}{rgb}{1.000000, 1.000000, 1.000000}
\pgfsetfillcolor{diafillcolor}
\pgfsetfillopacity{1.000000}
\pgfpathellipse{\pgfpoint{31.949958\du}{17.483992\du}}{\pgfpoint{3.949958\du}{0\du}}{\pgfpoint{0\du}{2.683992\du}}
\pgfusepath{fill}
\definecolor{dialinecolor}{rgb}{0.000000, 0.000000, 0.000000}
\pgfsetstrokecolor{dialinecolor}
\pgfsetstrokeopacity{1.000000}
\pgfpathellipse{\pgfpoint{31.949958\du}{17.483992\du}}{\pgfpoint{3.949958\du}{0\du}}{\pgfpoint{0\du}{2.683992\du}}
\pgfusepath{stroke}
% setfont left to latex
\definecolor{dialinecolor}{rgb}{0.000000, 0.000000, 0.000000}
\pgfsetstrokecolor{dialinecolor}
\pgfsetstrokeopacity{1.000000}
\definecolor{diafillcolor}{rgb}{0.000000, 0.000000, 0.000000}
\pgfsetfillcolor{diafillcolor}
\pgfsetfillopacity{1.000000}
\node[anchor=base,inner sep=0pt, outer sep=0pt,color=dialinecolor] at (31.949958\du,17.728958\du){Continue Normally};
\pgfsetlinewidth{0.100000\du}
\pgfsetdash{}{0pt}
\pgfsetbuttcap
{
\definecolor{diafillcolor}{rgb}{0.000000, 0.000000, 0.000000}
\pgfsetfillcolor{diafillcolor}
\pgfsetfillopacity{1.000000}
% was here!!!
\definecolor{dialinecolor}{rgb}{0.000000, 0.000000, 0.000000}
\pgfsetstrokecolor{dialinecolor}
\pgfsetstrokeopacity{1.000000}
\draw (24.716458\du,20.751228\du)--(28.615202\du,18.990238\du);
}
\definecolor{dialinecolor}{rgb}{0.000000, 0.000000, 0.000000}
\pgfsetstrokecolor{dialinecolor}
\pgfsetstrokeopacity{1.000000}
\draw (24.716458\du,20.751228\du)--(28.057637\du,19.242080\du);
\pgfsetlinewidth{0.100000\du}
\pgfsetdash{}{0pt}
\pgfsetmiterjoin
\definecolor{diafillcolor}{rgb}{1.000000, 1.000000, 1.000000}
\pgfsetfillcolor{diafillcolor}
\pgfsetfillopacity{1.000000}
\fill (28.160546\du,19.469917\du)--(28.513310\du,19.036261\du)--(27.954727\du,19.014244\du)--cycle;
\definecolor{dialinecolor}{rgb}{0.000000, 0.000000, 0.000000}
\pgfsetstrokecolor{dialinecolor}
\pgfsetstrokeopacity{1.000000}
\draw (28.160546\du,19.469917\du)--(28.513310\du,19.036261\du)--(27.954727\du,19.014244\du)--cycle;
\end{tikzpicture}
}
	\caption{%
		The overall design of the modified Android application Riot, which spawns Dendrite once it starts.
	}%
	\label{fig:design_overview}
\end{figure}

Once the user has been registered on the embedded Dendrite server, Riot stores the address of Dendrite along with the registration information.
Typically this makes sense, because the Matrix Standard assumes that server addresses do not change.
In our case, however, this does not apply.
Riot therefore only uses the address provided by Dendrite after initialization.
