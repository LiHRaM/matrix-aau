\chapter{Design}
This chapter consists of an initial section describing the constraints of the design, followed by a dilemma, and concluding with a design overview.
The dilemma motivates a choice of technologies which impacts the design.
Finally, we conclude the chapter by providing the final design overview.

\section{Constraints}
\todo{Describe what we can and can't do, e.g.~we can't change the Matrix protocol too much, can we?}

% \paragraph{New vs.~Old}
% The first two dilemmas are a result of working in open source communities.
% The code used in this project is constantly updated and refactored by contributors.
% Additionally, there is an effort to move development from Synapse and Riot to Dendrite and RiotX, and described in sections~\ref{sec:official_clients} and~\ref{sec:official_homeservers}.
% We therefore consider the advantages and disadvantages of choosing the new or the old in terms of homeserver and client.

\section{Dilemma: Stable versus Bleeding Edge}
The first dilemma consists of choosing a homeserver and a client to work with.
We must consider the advantages and disadvantages of both the older, more stable Synapse and Riot versus the newer Dendrite and RiotX.
There are many other clients and homeservers which we could consider as well, many of which are listed on the Matrix website.
However, as described in the analysis, these alternatives are not officially supported and are therefore risky choices.

\subsection{Approach}
We begin by considering the two homeservers described in Section~\ref{sec:official_homeservers}.
Afterwards we repeat the procedure for the two clients described in Section~\ref{sec:official_clients}.
We estimate the work required for each homeserver to support the minimum requirements for the project, which are:
\begin{itemize}
	\item{
	      \textbf{Peer-to-Peer}:
	      The server must be able to participate in a peer-to-peer network, by peer discovery and signaling its presence.
	      }
	\item{
	      \textbf{Android Packaging}:
	      Riot and RiotX use the Gradle\footnote{https://gradle.org} build system, which transforms source program into an Android application.
	      We therefore need a method to integrate the homeserver in this build system in order to package it with the application.
	      This also allows the clients to start and stop the servers as necessary.
	      }
\end{itemize}

Table~\ref{tab:homeserver_comparison} describes the result of this comparison, which we now describe in further detail.

\begin{table}[t]
	\center{}
	\begin{tabular}{lcc}
		\textbf{Server} & \multicolumn{1}{l}{\textbf{Peer-to-Peer}} & \multicolumn{1}{l}{\textbf{Android Packaging}} \\ \toprule
		Synapse         & No                                        & No                                             \\
		Dendrite        & Yes                                       & Yes
	\end{tabular}
	\caption{An overview of the fitness of the two homeservers for use as embedded servers.}\label{tab:homeserver_comparison}
\end{table}

\subsubsection{Synapse}
Synapse is a natural choice for many projects which need a Matrix homeserver.
It is the official, recommended homeserver for personal and commercial use, and is thoroughly tested.
It is the reference implementation, and is stable and feature complete.
On the other hand, it does not support either peer-to-peer networking or Android packaging.
If we were to choose Synapse, we would have to implement all the peer-to-peer logic which is already implemented in Dendrite.
Additionally, as there are no Python implementations of libp2p, Synapse would have to use a different set of libraries.
One could also choose to invest in libp2p for Python and implement those libraries as well, but that would require an even greater amount of work.
In terms of Android packaging, there are no official solutions for Python, but there is an unofficial solution known as BeeWare\footnotemark.
\footnotetext{https://beeware.org}
Unfortunately, as its documentation states\footnotemark, the build tools are incompatible with the latest changes in the Android ecosystem.
\footnotetext{https://docs.beeware.org/en/latest/tutorial/tutorial-5/index.html}
In summary, it would be a major effort to use Synapse as the embedded homeserver at this time.

\subsubsection{Dendrite}
Dendrite is not recommended for serious use yet.
It lacks the stability and features that Synapse has.
For example, it only supports Client-Server API up to version v0.3.0, whereas the latest stable is v0.6.0.
It is actively being changed, and is overall a very new project compared to the older and more stable Synapse.
However, through the \texttt{gomobile} project, it can be packaged for Android, and there is already a version which contains a peer-to-peer implementation.
Therefore, although Dendrite lacks the stability and completeness of Synapse, it would require much less work to function as an embedded, peer-to-peer homeserver than Synapse would.

\subsubsection{Riot}
Riot for Android is conceptually similar to Synapse, in the sense that it is an older, more stable and mature application than its newer alternative.
It supports both Synapse and Dendrite.
However, it is not under active development anymore, as described in Section~\ref{sec:official_clients}.
Instead, they point new contributors to RiotX instead, which is now the main focus for Android development.

\subsubsection{RiotX}
RiotX is similar to Dendrite, in that it is the official next generation client.
Currently, it is in beta stage, and does not support all the features that Riot does.
Notably, it does not support older versions of the Client-Server API, requiring versions v0.5.0 and v0.6.0.
This makes it incompatible with Dendrite at this time.
Choosing RiotX would therefore require work to bring Dendrite up to those versions.

\subsection{Resolution}
We choose Dendrite as the embedded homeserver, based on the criteria listed in Table~\ref{tab:homeserver_comparison}, and Riot for Android as the client, as RiotX does not at this time support Dendrite as an embedded homeserver.

\section{Design Overview}

\todo{High-level overview}
\todo{More detailed selected components}
\todo{Complete design of my additions}