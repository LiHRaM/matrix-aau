\chapter*{Preface}
\addcontentsline{toc}{chapter}{\protect\numberline{}Preface}
This student report is a part of the Mobility semester project of the Software Engineering Master's degree program at Aalborg University.
It was written under the supervision of Ivan Aaen.

\section*{Git}
\addcontentsline{toc}{section}{\protect\numberline{}Git}

We assume knowledge with git\cite{git}, which is a version control system for code.
In git, code is grouped into repositories, which are snapshots of files and folders at a certain point in time.
These snapshots are also known as commits, and are identified by a hash.
These commits form the version history of a git repository.
The reader should be familiar with the basic terminology of git, and understand terms such as a branch, a merge and a pull request.


\section*{\faGithub~GitHub}
\addcontentsline{toc}{section}{\protect\numberline{}GitHub}
GitHub is an online environment for managing git repositories.
The code associated with this report is stored on GitHub.
Referring to repositories, issues, pull requests and branches can become quite repetitive.
In order to make it easier to write and read these references, we use the following shorthand to refer to GitHub repositories:
\begin{itemize}
      \item{
            \textbf{Repository}: \github{org/repository}\\
            \textit{Before}: https://github.com/lihram/aau\\
            \textit{After}: \github{lihram/aau}
            }
      \item{
            \textbf{Issue or Pull Request}: \github{org/repository\#ID}\\
            \textit{Before}: \url{https://github.com/lihram/aau/pull/1}\\
            \textit{After}: \github{lihram/aau\#1}\\
            Several issues or pull requests can be lumped together, with multiple IDs separated by a comma: \github{org/repository\#1,2,3}.
            }
      \item{
            \textbf{Branch}: \github{org/repository/branch}\\
            \textit{Before}: \url{https://github.com/LiHRaM/aau/tree/gh-pages}\\
            \textit{After}: \github{lihram/aau/gh-pages}
            }
\end{itemize}

\listoftodos
