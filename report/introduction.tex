\chapter{Introduction}
In 1971, the first email was sent\cite{tomlinson2009first}.
Email is implemented using decentralized protocols for sending simple messages between recipients, specified for example in the \ac{SMTP} \ac{RFC}\cite{RFC5321}.
One may register with an email provider, and write messages to any other provider implementing the same protocols, as long as they have the recipient's address, known as an email address.
Mobile providers similarly allow people to contact others outside of their network.
This is an elegant, scalable and privacy-respecting architecture.
However, there's been a growing trend of \textit{walled gardens}\cite{walled_gardens_gunnar_wolf_acm_2018}, where users' communication is restricted to a single provider, or service.
Three examples of such services are Slack, WhatsApp and Facebook Messenger, with WhatsApp announcing a userbase exceeding two billion users, and Facebook Messenger 1.3 billion users\cite{walled_gardens_gunnar_wolf_acm_2018,whatsapp_2b_users_archive_org,messenger_1pt3b_users}.

\section{Initial Problem}\label{subsec:initial_problem_statement}
Closed-platform services, or walled gardens, are a fragmenting influence on peoples' communication.
Although these services run over the internet, they are designed in isolation, and do not adhere to an external standard.
In order to use such a service, a user must acquire a client designed with a private procotol.
This has the negative effect of people having to keep track of multiple services for the same task, i.e.~communication.
They are forced to rely on external vendors for the storage and security of their messages.
We refer to this problem as \textit{the walled garden problem}.

\begin{figure}
    \centering
    \resizebox{0.5\linewidth}{!}{% Graphic for TeX using PGF
% Title: /home/lihram/git/github.com/lihram/p8-matrix-p2p/report/graphics/client-server.dia
% Creator: Dia v0.97.3
% CreationDate: Tue Mar 31 12:48:39 2020
% For: lihram
% \usepackage{tikz}
% The following commands are not supported in PSTricks at present
% We define them conditionally, so when they are implemented,
% this pgf file will use them.
\ifx\du\undefined
  \newlength{\du}
\fi
\setlength{\du}{15\unitlength}
\begin{tikzpicture}
\pgftransformxscale{1.000000}
\pgftransformyscale{-1.000000}
\definecolor{dialinecolor}{rgb}{0.000000, 0.000000, 0.000000}
\pgfsetstrokecolor{dialinecolor}
\definecolor{dialinecolor}{rgb}{1.000000, 1.000000, 1.000000}
\pgfsetfillcolor{dialinecolor}
\pgfsetlinewidth{0.100000\du}
\pgfsetdash{}{0pt}
\pgfsetdash{}{0pt}
\pgfsetbuttcap
\pgfsetmiterjoin
\pgfsetlinewidth{0.100000\du}
\pgfsetbuttcap
\pgfsetmiterjoin
\pgfsetdash{}{0pt}
\definecolor{dialinecolor}{rgb}{1.000000, 1.000000, 1.000000}
\pgfsetfillcolor{dialinecolor}
\pgfpathmoveto{\pgfpoint{4.257143\du}{0.800000\du}}
\pgfpathlineto{\pgfpoint{10.542857\du}{0.800000\du}}
\pgfpathlineto{\pgfpoint{11.800000\du}{2.875000\du}}
\pgfpathlineto{\pgfpoint{10.542857\du}{4.950000\du}}
\pgfpathlineto{\pgfpoint{4.257143\du}{4.950000\du}}
\pgfpathlineto{\pgfpoint{3.000000\du}{2.875000\du}}
\pgfpathlineto{\pgfpoint{4.257143\du}{0.800000\du}}
\pgfusepath{fill}
\definecolor{dialinecolor}{rgb}{0.000000, 0.000000, 0.000000}
\pgfsetstrokecolor{dialinecolor}
\pgfpathmoveto{\pgfpoint{4.257143\du}{0.800000\du}}
\pgfpathlineto{\pgfpoint{10.542857\du}{0.800000\du}}
\pgfpathlineto{\pgfpoint{11.800000\du}{2.875000\du}}
\pgfpathlineto{\pgfpoint{10.542857\du}{4.950000\du}}
\pgfpathlineto{\pgfpoint{4.257143\du}{4.950000\du}}
\pgfpathlineto{\pgfpoint{3.000000\du}{2.875000\du}}
\pgfpathlineto{\pgfpoint{4.257143\du}{0.800000\du}}
\pgfusepath{stroke}
% setfont left to latex
\definecolor{dialinecolor}{rgb}{0.000000, 0.000000, 0.000000}
\pgfsetstrokecolor{dialinecolor}
\node at (7.400000\du,3.069043\du){Server};
\definecolor{dialinecolor}{rgb}{1.000000, 1.000000, 1.000000}
\pgfsetfillcolor{dialinecolor}
\pgfpathellipse{\pgfpoint{2.996636\du}{8.998318\du}}{\pgfpoint{2.353364\du}{0\du}}{\pgfpoint{0\du}{2.351682\du}}
\pgfusepath{fill}
\pgfsetlinewidth{0.100000\du}
\pgfsetdash{}{0pt}
\pgfsetdash{}{0pt}
\pgfsetmiterjoin
\definecolor{dialinecolor}{rgb}{0.000000, 0.000000, 0.000000}
\pgfsetstrokecolor{dialinecolor}
\pgfpathellipse{\pgfpoint{2.996636\du}{8.998318\du}}{\pgfpoint{2.353364\du}{0\du}}{\pgfpoint{0\du}{2.351682\du}}
\pgfusepath{stroke}
% setfont left to latex
\definecolor{dialinecolor}{rgb}{0.000000, 0.000000, 0.000000}
\pgfsetstrokecolor{dialinecolor}
\node at (2.996636\du,9.192371\du){Client A};
\definecolor{dialinecolor}{rgb}{1.000000, 1.000000, 1.000000}
\pgfsetfillcolor{dialinecolor}
\pgfpathellipse{\pgfpoint{11.902500\du}{9.025000\du}}{\pgfpoint{2.447500\du}{0\du}}{\pgfpoint{0\du}{2.425000\du}}
\pgfusepath{fill}
\pgfsetlinewidth{0.100000\du}
\pgfsetdash{}{0pt}
\pgfsetdash{}{0pt}
\pgfsetmiterjoin
\definecolor{dialinecolor}{rgb}{0.000000, 0.000000, 0.000000}
\pgfsetstrokecolor{dialinecolor}
\pgfpathellipse{\pgfpoint{11.902500\du}{9.025000\du}}{\pgfpoint{2.447500\du}{0\du}}{\pgfpoint{0\du}{2.425000\du}}
\pgfusepath{stroke}
% setfont left to latex
\definecolor{dialinecolor}{rgb}{0.000000, 0.000000, 0.000000}
\pgfsetstrokecolor{dialinecolor}
\node at (11.902500\du,9.219053\du){Client B};
\pgfsetlinewidth{0.100000\du}
\pgfsetdash{{\pgflinewidth}{0.200000\du}}{0cm}
\pgfsetdash{{\pgflinewidth}{0.200000\du}}{0cm}
\pgfsetbuttcap
{
\definecolor{dialinecolor}{rgb}{0.000000, 0.000000, 0.000000}
\pgfsetfillcolor{dialinecolor}
% was here!!!
\definecolor{dialinecolor}{rgb}{0.000000, 0.000000, 0.000000}
\pgfsetstrokecolor{dialinecolor}
\draw (11.118780\du,6.676105\du)--(10.542857\du,4.950000\du);
}
\pgfsetlinewidth{0.100000\du}
\pgfsetdash{{\pgflinewidth}{0.200000\du}}{0cm}
\pgfsetdash{{\pgflinewidth}{0.200000\du}}{0cm}
\pgfsetbuttcap
{
\definecolor{dialinecolor}{rgb}{0.000000, 0.000000, 0.000000}
\pgfsetfillcolor{dialinecolor}
% was here!!!
\definecolor{dialinecolor}{rgb}{0.000000, 0.000000, 0.000000}
\pgfsetstrokecolor{dialinecolor}
\draw (4.257143\du,4.950000\du)--(3.710749\du,6.704831\du);
}
\end{tikzpicture}
}
    \caption{
        Typical client-server architecture.
        Client A and Client B communicate via a Server, which is responsible for passing on the messages.
    }
    \label{fig:client_server}
\end{figure}

\subsection{Trust}\label{sec:Trust}
One of the main drawbacks of using a walled garden service is \textit{trust}.
Client-server architecture places servers as mediators between clients, forcing all communication to go through servers, as illustrated in figure \ref{fig:client_server}.
A common feature of communications servers is that the entirety of messages be available for an indefinite period of time.
Platforms which follow this architecture for messaging therefore store the messages of all participants.
For closed-source vendors, it is intractable to know whether the messages passed are done so securely and privately.
Users are forced to either rely on the word of the service providers, or use an alternative.
In the cases of WhatsApp and Facebook Messenger, several such features are advertised, but regardless not verified\cite{twitter_comms_protocol_comparison}.
Another example is the Zoom service, which has recently been exposed for lying about having end-to-end encryption\cite{zoom_e2ee_or_not}.
Trust is an important factor in online communication.
Any proposed solution to the walled garden problem must account for that aspect.

\subsection{Ain't No Such Thing}
The phrase \textit{There ain't no such thing as a free lunch} is a popular adage which means that everything has a cost.
This cost may not always be immediately apparent, especially when something is advertised as free.
Consider for example Facebook, which doesn't cost money.
Facebook gathers vast information on its users, which can be used for various purposes, such as targeted advertising.
In this sense, the users pay for their free lunch with their privacy.
This practice has been abused maliciously, for example by Cambridge Analytica,
who harvested information on millions of people through a security flaw in the Facebook platform and used it to influence the 2016 U.S.~elections\cite{cadwalladr2018revealed, cadwalladr2018cambridge, isaak2018user, berghel2018malice}.
This raises two issues, namely \textit{cost} and \textit{privacy}.

\section{Matrix}
The Matrix standard is a potential solution to the \textit{walled garden problem}, and is described as an "open standard for interoperable, decentralised, real-time communication over IP"\cite{matrix_org}.
On the choice of the name Matrix, the Matrix organization states: "We are called Matrix because we provide a structure in which all communication can be matrixed together."\cite{matrix_org_faq}

\begin{figure}
    \centering
    \resizebox{0.5\linewidth}{!}{
        Matrix = \bordermatrix{
            ~      & Work & Friends & Family & Hobbies\cr
            Msg       & 1 & 1 & 1 & 1\cr
            Voice     & 1 & 1 & 1 & 1\cr
            Video     & 1 & 1 & 1 & 1\cr
            Bots      & 1 & 1 & 1 & 1\cr
            Bridges   & 1 & 1 & 1 & 1\cr
        }
    }
    \caption{Matrix provides the fundamental structure for matrixing together all communication.}
    \label{fig:matrixing_together}
\end{figure}

The Matrix open standard\cite{matrix_org_spec} is a specification for a network of independent servers which may communicate.
It supports recent features, such as live messaging, calls, and video conferencing.
Figure \ref{fig:matrix_structure} demonstrates how two users might be connected via the Matrix protocol.
There are two servers: \textit{matrix.a.com}, and \textit{matrix.b.com}.
The users are \textit{@a:a.com}, and \textit{@b:b.com}.
A message from \textit{@a:a.com} would pass through both of the servers before \textit{@b:b.com} would receive them.
This is an implementation of the client-server architecture with a decentralized approach, which tackles the issues of trust and privacy.
As servers can be owned and operated by anyone, users are not required to trust a third party.
For the most extreme privacy concerns, a user might simply host a private server which only permitted internal communication, and required encryption on all channels.
For more casual use cases, the network of servers provides redundancy, as messages are copied by each participating homeserver.

\begin{figure}
    \centering
    \resizebox{0.5\linewidth}{!}{% Graphic for TeX using PGF
% Title: /home/lihram/git/github.com/lihram/p8-matrix-p2p/report/graphics/matrix.dia
% Creator: Dia v0.97.3
% CreationDate: Thu Apr  2 21:27:29 2020
% For: lihram
% \usepackage{tikz}
% The following commands are not supported in PSTricks at present
% We define them conditionally, so when they are implemented,
% this pgf file will use them.
\ifx\du\undefined
  \newlength{\du}
\fi
\setlength{\du}{15\unitlength}
\begin{tikzpicture}
\pgftransformxscale{1.000000}
\pgftransformyscale{-1.000000}
\definecolor{dialinecolor}{rgb}{0.000000, 0.000000, 0.000000}
\pgfsetstrokecolor{dialinecolor}
\definecolor{dialinecolor}{rgb}{1.000000, 1.000000, 1.000000}
\pgfsetfillcolor{dialinecolor}
\definecolor{dialinecolor}{rgb}{1.000000, 1.000000, 1.000000}
\pgfsetfillcolor{dialinecolor}
\pgfpathellipse{\pgfpoint{23.159091\du}{11.454516\du}}{\pgfpoint{2.187187\du}{0\du}}{\pgfpoint{0\du}{1.975319\du}}
\pgfusepath{fill}
\pgfsetlinewidth{0.100000\du}
\pgfsetdash{}{0pt}
\pgfsetdash{}{0pt}
\pgfsetmiterjoin
\definecolor{dialinecolor}{rgb}{0.000000, 0.000000, 0.000000}
\pgfsetstrokecolor{dialinecolor}
\pgfpathellipse{\pgfpoint{23.159091\du}{11.454516\du}}{\pgfpoint{2.187187\du}{0\du}}{\pgfpoint{0\du}{1.975319\du}}
\pgfusepath{stroke}
% setfont left to latex
\definecolor{dialinecolor}{rgb}{0.000000, 0.000000, 0.000000}
\pgfsetstrokecolor{dialinecolor}
\node at (23.159091\du,11.648578\du){@a:a.com};
\definecolor{dialinecolor}{rgb}{1.000000, 1.000000, 1.000000}
\pgfsetfillcolor{dialinecolor}
\pgfpathellipse{\pgfpoint{23.257966\du}{18.985725\du}}{\pgfpoint{2.191191\du}{0\du}}{\pgfpoint{0\du}{2.039375\du}}
\pgfusepath{fill}
\pgfsetlinewidth{0.100000\du}
\pgfsetdash{}{0pt}
\pgfsetdash{}{0pt}
\pgfsetmiterjoin
\definecolor{dialinecolor}{rgb}{0.000000, 0.000000, 0.000000}
\pgfsetstrokecolor{dialinecolor}
\pgfpathellipse{\pgfpoint{23.257966\du}{18.985725\du}}{\pgfpoint{2.191191\du}{0\du}}{\pgfpoint{0\du}{2.039375\du}}
\pgfusepath{stroke}
% setfont left to latex
\definecolor{dialinecolor}{rgb}{0.000000, 0.000000, 0.000000}
\pgfsetstrokecolor{dialinecolor}
\node at (23.257966\du,19.179788\du){@b:b.com};
\definecolor{dialinecolor}{rgb}{1.000000, 1.000000, 1.000000}
\pgfsetfillcolor{dialinecolor}
\fill (30.775500\du,9.312650\du)--(30.775500\du,13.262650\du)--(36.525500\du,13.262650\du)--(36.525500\du,9.312650\du)--cycle;
\pgfsetlinewidth{0.100000\du}
\pgfsetdash{}{0pt}
\pgfsetdash{}{0pt}
\pgfsetmiterjoin
\definecolor{dialinecolor}{rgb}{0.000000, 0.000000, 0.000000}
\pgfsetstrokecolor{dialinecolor}
\draw (30.775500\du,9.312650\du)--(30.775500\du,13.262650\du)--(36.525500\du,13.262650\du)--(36.525500\du,9.312650\du)--cycle;
% setfont left to latex
\definecolor{dialinecolor}{rgb}{0.000000, 0.000000, 0.000000}
\pgfsetstrokecolor{dialinecolor}
\node at (33.650500\du,11.481703\du){matrix.b.com};
\definecolor{dialinecolor}{rgb}{1.000000, 1.000000, 1.000000}
\pgfsetfillcolor{dialinecolor}
\fill (30.763900\du,17.046800\du)--(30.763900\du,20.996800\du)--(36.513900\du,20.996800\du)--(36.513900\du,17.046800\du)--cycle;
\pgfsetlinewidth{0.100000\du}
\pgfsetdash{}{0pt}
\pgfsetdash{}{0pt}
\pgfsetmiterjoin
\definecolor{dialinecolor}{rgb}{0.000000, 0.000000, 0.000000}
\pgfsetstrokecolor{dialinecolor}
\draw (30.763900\du,17.046800\du)--(30.763900\du,20.996800\du)--(36.513900\du,20.996800\du)--(36.513900\du,17.046800\du)--cycle;
% setfont left to latex
\definecolor{dialinecolor}{rgb}{0.000000, 0.000000, 0.000000}
\pgfsetstrokecolor{dialinecolor}
\node at (33.638900\du,19.215853\du){matrix.a.com};
\pgfsetlinewidth{0.100000\du}
\pgfsetdash{}{0pt}
\pgfsetdash{}{0pt}
\pgfsetbuttcap
{
\definecolor{dialinecolor}{rgb}{0.000000, 0.000000, 0.000000}
\pgfsetfillcolor{dialinecolor}
% was here!!!
\pgfsetarrowsend{stealth}
\definecolor{dialinecolor}{rgb}{0.000000, 0.000000, 0.000000}
\pgfsetstrokecolor{dialinecolor}
\draw (25.854118\du,11.570902\du)--(30.301569\du,11.593827\du);
}
\pgfsetlinewidth{0.100000\du}
\pgfsetdash{}{0pt}
\pgfsetdash{}{0pt}
\pgfsetbuttcap
{
\definecolor{dialinecolor}{rgb}{0.000000, 0.000000, 0.000000}
\pgfsetfillcolor{dialinecolor}
% was here!!!
\pgfsetarrowsend{stealth}
\definecolor{dialinecolor}{rgb}{0.000000, 0.000000, 0.000000}
\pgfsetstrokecolor{dialinecolor}
\draw (33.488145\du,13.966565\du)--(33.511070\du,16.671716\du);
}
\pgfsetlinewidth{0.100000\du}
\pgfsetdash{}{0pt}
\pgfsetdash{}{0pt}
\pgfsetbuttcap
{
\definecolor{dialinecolor}{rgb}{0.000000, 0.000000, 0.000000}
\pgfsetfillcolor{dialinecolor}
% was here!!!
\pgfsetarrowsend{stealth}
\definecolor{dialinecolor}{rgb}{0.000000, 0.000000, 0.000000}
\pgfsetstrokecolor{dialinecolor}
\draw (30.003544\du,19.468567\du)--(25.739493\du,19.491492\du);
}
\pgfsetlinewidth{0.100000\du}
\pgfsetdash{}{0pt}
\pgfsetdash{}{0pt}
\pgfsetbuttcap
{
\definecolor{dialinecolor}{rgb}{0.000000, 0.000000, 0.000000}
\pgfsetfillcolor{dialinecolor}
% was here!!!
\pgfsetarrowsend{stealth}
\definecolor{dialinecolor}{rgb}{0.000000, 0.000000, 0.000000}
\pgfsetstrokecolor{dialinecolor}
\draw (25.762418\du,18.505716\du)--(30.232794\du,18.528641\du);
}
\pgfsetlinewidth{0.100000\du}
\pgfsetdash{}{0pt}
\pgfsetdash{}{0pt}
\pgfsetbuttcap
{
\definecolor{dialinecolor}{rgb}{0.000000, 0.000000, 0.000000}
\pgfsetfillcolor{dialinecolor}
% was here!!!
\pgfsetarrowsend{stealth}
\definecolor{dialinecolor}{rgb}{0.000000, 0.000000, 0.000000}
\pgfsetstrokecolor{dialinecolor}
\draw (34.473920\du,16.465391\du)--(34.450995\du,13.829015\du);
}
\pgfsetlinewidth{0.100000\du}
\pgfsetdash{}{0pt}
\pgfsetdash{}{0pt}
\pgfsetbuttcap
{
\definecolor{dialinecolor}{rgb}{0.000000, 0.000000, 0.000000}
\pgfsetfillcolor{dialinecolor}
% was here!!!
\pgfsetarrowsend{stealth}
\definecolor{dialinecolor}{rgb}{0.000000, 0.000000, 0.000000}
\pgfsetstrokecolor{dialinecolor}
\draw (29.934769\du,10.573665\du)--(25.739493\du,10.550740\du);
}
\end{tikzpicture}
}
    \caption{
        Matrix allows users to communicate across servers.
        Using federation, each server keeps an updated copy of metadata created by its clients.
        People are encouraged to run their own Matrix servers, for increased privacy and performance.
    }
    \label{fig:matrix_structure}
\end{figure}

The standard has been implemented by several clients\cite{matrix_org_clients}.
There is ongoing work on implementing SDKs\cite{matrix_org_sdks}, which provide the foundational logic for implementing more applications.
Through bridges, Matrix allows communicating with different protocols\cite{matrix_org_bridges}, such as WhatsApp or Messenger.
The reference implementation of the Matrix open standard is Synapse\cite{matrix_org_synapse}, which is written in Python.
The standard was declared stable and released as 1.0 on June 10th, 2019\cite{matrix_org_spec}, and is still very young.
Its limitations become apparent as one considers its applications, such as \ac{P2P} Matrix.

\subsection{Peer-to-Peer Networks}
In contrast to the client-server model described in section \ref{sec:Trust}, a peer-to-peer network allows nodes to connect directly to each other.
There is no central authority, and all nodes as such have equal privilege and functionality.
Nodes are apty named peers.
This model is beneficial in terms of cost, as it relies on the collective computing power of its peers instead of a dedicated service.
It also has the potential to increase the privacy of its users, due to its more direct nature.


\subsection{Peer-to-Peer Matrix}
The Matrix team is working on Dendrite\cite{matrix_org_dendrite}, an experimental server written in the Go programming language, which is the base for \ac{P2P} experiments.
This approach overcomes some of the drawbacks of the current implementation of Matrix, which requires each user to be registered with a homeserver with a \ac{DNS} entry and a static IP address.
This requirement causes several issues:
\begin{itemize}
    \item{
          \textbf{Domain Knowledge}:
          Server management requires technical knowledge.
          }
    \item{
          \textbf{Cost}:
          Some solutions require less knowledge, but can be expensive.
          }
    \item{
          \textbf{Inflexibility}:
          A server cannot be migrated between \ac{DNS} entries, i.e.~it cannot change names.
          }
\end{itemize}
Instead, \ac{P2P} Matrix runs homeservers on-device.
This solution does not have to cost money, and does not require knowledge of server management.
Users do not have to sign up with an external homeserver.
The development of \ac{P2P} Matrix is currently experimental and ongoing.
Its application is currently limited to the web at \url{https://p2p.riot.im}, but could expand into other fields such as embedded, or mobile devices.
Since users host their own servers on-device, the hosting is free, and might compete with the free services provided by walled gardens.

\subsection{Smartphones}
Due to the popularity and widespread use of mobile phones, experimentation with \ac{P2P} Matrix on smartphones would be an interesting task.
\textit{Besides an experimental browser-based solution\cite{fosdem_event_p2p_matrix}, there is no \ac{P2P} Matrix application for mobile clients, such as smartphones.}

The benefits of having an on-device homeserver for a telephone would have strong benefits for users, such as:
\begin{itemize}
    \item Full control of data,
    \item No registration necessary, and
    \item Increased privacy.
\end{itemize}

\noindent
Modern mobile devices have software which monitors the battery use of applications, and warns the user if applications exceed the normal range.
To accomodate this limitation, considerations must be in the use of processors, memory and other resources.

\begin{center}
    We propose the following initial problem: \textit{What can we do to get peer-to-peer Matrix on the smartphone, while respecting the constraints of mobile phones, such as battery life and memory usage?}
\end{center}

