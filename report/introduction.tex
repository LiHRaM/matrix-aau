\section{Introduction}
In 1971, the first email was sent\cite{tomlinson2009first}.
Email is currently implemented using decentralized protocols for sending simple messages between recipients, specified for example in the SMTP RFC\cite{RFC5321}.
One may register with an email provider, and write messages to any other provider implementing the same protocols, as long as they have the recipient's address, known as an email address.
Mobile providers similarly allow people to contact others outside of their network.
However, there's been a growing trend of \textit{communication silos}, where users may only contact other users of the same service.
Examples of such services are WhatsApp, which has announced that its userbase has now exceeded two billion users\cite{whatsapp_2b_users_archive_org}, and Facebook Messenger, which similarly reported having 1.3 billion users\cite{messenger_1pt3b_users}.

\paragraph{Communication Silos}
Although these services run over the internet, they are designed in isolation, and do not adhere to an external standard.
In order to use such a service, a user must acquire a client designed for this private procotol.
This approach does not scale well in terms of battery or network traffic, as each application separately performs similar routine actions, such as checking for and sending messages.
We refer to the challenges of communication silos as \textit{the silo problem}.

\paragraph{Matrix}
The Matrix open standard\cite{matrix_org_spec} is a specification for a network similar in structure to email.
It supports more recent features, such as live messaging, calls, and video conferencing.
This standard is a potential solution to \textit{the silo problem}, and has several strong points.
It is designed for interoperability.
It has multiple clients\cite{matrix_org_clients} and SDKs\cite{matrix_org_sdks}.
Through bridges, Matrix allows communicating with different protocols\cite{matrix_org_bridges}.
The reference implementation of the Matrix open standard is Synapse\cite{matrix_org_synapse}, which is written mostly in Python.

\subsection{Initial Problem Statement}\label{subsec:initial_problem_statement}
We will explore the current use of the Matrix open standard and its current limitations.
We might consider alternatives to current solutions, as well as novel solutions to open problems.

We propose the following initial problem statement:
\begin{itemize}
    \item How is the Matrix open standard currently being used?
    \item What are the limitations of the standard?
    \item What are the limitations of the reference implementation?
    \item What are the necessary properties of solutions to these limitations?
\end{itemize}