\section{Introduction}
In 1971, the first email was sent\cite{tomlinson2009first}.
Email is currently implemented using decentralized protocols for sending simple messages between recipients, specified for example in the SMTP RFC\cite{RFC5321}.
One may register with an email provider, and write messages to any other provider implementing the same protocols, as long as they have the recipient's address, known as an email address.
Mobile providers similarly allow people to contact others outside of their network.
However, there's been a growing trend of \textit{communication silos}, where users may only contact other users of the same service.
Examples of such services are WhatsApp, which has announced that its userbase has now exceeded two billion users\cite{whatsapp_2b_users_archive_org}, and Facebook Messenger, which similarly reported having 1.3 billion users\cite{messenger_1pt3b_users}.

\subsection{Motivation}
\paragraph{Communication Silos}
Although these services run over the internet, they are designed in isolation, and do not adhere to an external standard.
In order to use such a service, a user must acquire a client designed for this private procotol.
This approach does not scale well in terms of battery or network traffic, as each application separately performs similar routine actions, such as checking for and sending messages.
We refer to the challenges of communication silos as \textit{the silo problem}.

\paragraph{Security}
In the typical client-server architecture, the client is forced to operate with some level of trust.
In the case of email, the original architecture required that both participants be online, but has now adapted such that some server just needs to receive it and be able to pass it on to the recipient when possible\cite{tomlinson2009first}.
Platforms which follow this architecture for messaging therefore store the messages of all conversation participants.
The storage format and security depends on the software run on the server, which receives, stores and passes the messages on.
For closed-source vendors, it is intractable to know whether the messages passed are done so securely, or whether third parties can access the data.
Users are therefore forced to either rely on the word of the service providers, or use an alternative.
In the cases of WhatsApp and Facebook Messenger, several such features are boasted, but regardless not verified\cite{twitter_comms_protocol_comparison}.

\subsection{Initial Problem Statement}\label{subsec:initial_problem_statement}
\paragraph{Matrix}
The Matrix open standard\cite{matrix_org_spec} is a specification for a network similar in structure to email.
It supports more recent features, such as live messaging, calls, and video conferencing.
This standard is a potential solution to \textit{the silo problem}, and has several strong points.
It is designed for interoperability.
It has multiple clients\cite{matrix_org_clients} and SDKs\cite{matrix_org_sdks}.
Through bridges, Matrix allows communicating with different protocols\cite{matrix_org_bridges}.
The reference implementation of the Matrix open standard is Synapse\cite{matrix_org_synapse}, which is written mostly in Python.
The Matrix team is also working on Dendrite\cite{matrix_org_dendrite}, an experimental server written in Golang, which is the base for peer-to-peer experiments.

\paragraph{Dendrite and Peer-to-Peer Matrix}
Matrix follows a client-server architecture, where servers federate messages and are responsible for passing them on.
Peer discovery is done via DNS, which forces users to have a static IP address and a DNS entry for their homeserver.
This is not practical for privacy-minded users who aren't technologically savvy.
The Matrix team began experimenting with replacing the DNS peer discovery with a peer-to-peer equivalent, which would allow servers to run off of mobile devices such as laptops and phones, or even IoT devices.
They successfully embedded Dendrite as a WASM blob into their web client, and are currently continuing work on this front.

We propose the following initial problem statement:
\begin{itemize}
    \item How feasible is it to embed Dendrite into a mobile client?
    \item How does peer-to-peer Matrix perform on mobile devices?
\end{itemize}